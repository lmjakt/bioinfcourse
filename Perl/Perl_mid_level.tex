% Created 2015-05-28 Thu 13:42
\documentclass[11pt]{article}
\usepackage[utf8]{inputenc}
\usepackage[T1]{fontenc}
\usepackage{fixltx2e}
\usepackage{graphicx}
\usepackage{longtable}
\usepackage{float}
\usepackage{wrapfig}
\usepackage{rotating}
\usepackage[normalem]{ulem}
\usepackage{amsmath}
\usepackage{textcomp}
\usepackage{marvosym}
\usepackage{wasysym}
\usepackage{amssymb}
\usepackage{capt-of}
\usepackage{hyperref}
\tolerance=1000
\usepackage{minted}
\usepackage{color}
\usepackage{listings}
\usepackage{grffile}
\usepackage[inline]{enumitem}
\usepackage{setspace}
\usepackage{xcolor}
\hypersetup{
colorlinks,
linkcolor={red!50!black},
citecolor={blue!50!black},
urlcolor={blue!80!black}
}
\usepackage{setspace}%% The linestretch
\singlespacing
\usepackage[format=hang,indention=0cm,singlelinecheck=true,justification=raggedright,labelfont={normalsize,bf},textfont={normalsize}]{caption} %
\usepackage{vmargin}
\setpapersize{A4}
\setmarginsrb{2.5cm}{1cm}% links, oben
{2.5cm}{2cm}% rechts, unten
{12pt}{30pt}% Kopf: Höhe, Abstand
{12pt}{30pt}% Fuß: Höhe, AB
\usepackage{upquote}
%  use straight quotes when printing a command in minted
\AtBeginDocument{%
\def\PYZsq{\textquotesingle}%
}
\definecolor{mintedbackground}{rgb}{0.95,0.95,0.95}
\setlength{\parindent}{0pt}
\setlength{\parskip}{\baselineskip}
\definecolor{mintedbackground}{rgb}{0.95,0.95,0.95}

%% I detest indentation in footnotes etc, so try this:
\makeatletter
\renewcommand\@makefntext[1]{\noindent\makebox[0em][r]{\@makefnmark}\footnotesize#1}
\makeatother
%% the makeatletter and makeatother are required to allow me to
%% to change the macro beginning with an @. (though when I call it
%% I don't use the @ ... 

\renewcommand\scriptsize\normalsize

\author{Martin Jakt\thanks{University of Nordland, Norway}}
\date{\textbf{Bioinformatics \& Genomics}}
\title{\textbf{Intermediate level perl} (2015-09-11)}
\hypersetup{
 pdfkeywords={},
  pdfsubject={},
  pdfcreator={}}
\begin{document}

\maketitle
%\tableofcontents

\section{The story so far}
\label{sec-1}
By this stage you should be comfortable working with variables and arrays (also
known as vectors or lists) of values. You also have some idea of how you can
use 2-dimensional arrays to represent values in a matrix, and how to use a loop
to access and process (i.e. do something with) the values present in an array.
This knowledge is enough to solve a lot of problems, though, perhaps not in
the most elegant manner\footnote{Many would argue that there are no elegant solutions
in Perl. Please ignore such arguments}.

To enable you to solve problems efficiently with Perl we will introduce four new fields:
\begin{enumerate}
\item Regular expressions
\item File input output
\item Associative arrays and more complex data structures
\item Writing your own functions
\end{enumerate}

Of these 1 \& 2 are described in some detail in the previous introduction to Perl
and will only be briefly covered in this document. For more details please refer
to the first introduction. The third topic is useful for organising data in a
logical manner within your script. The fourth topic allows you to reuse 
small\footnote{well, they can be rather large} pieces of code by creating functions
that can be called from the main body of your script. This not only means
that you can do less typing but also lets you organise your code in a more readable
manner and reduces the number of errors. Functions also make it possible to do
funky things whereby a function calls itself until some condition is met. This is
called recursion and is a natural fit for quite a lot of computational problems but
understanding what happens during the execution of recursive functions can be
quite difficult to get your head around.

\subsection{Regular expressions}
\label{sec-1-1}
Regular expressions provide a way to summarise patterns of text. They are used widely
throughout 

\begin{minted}[fontsize=\scriptsize,bgcolor=lightgray,linenos]{perl}
## in this case we have an array of values @values
## that has been defined somewhere else in this script
## we call the mean() function on this and it returns the mean
## value. We can use this to assign to a new variable:

$mean = mean(@values);

## or we can make use of it in a conditional statement:

if( mean(@values) > 0 ){
  ## Do something clever here
}else{
  ## Do something different here
}
\end{minted}




\end{document}



