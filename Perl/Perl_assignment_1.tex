% Created 2015-05-28 Thu 13:42
\documentclass[11pt]{article}
\usepackage[utf8]{inputenc}
\usepackage[T1]{fontenc}
\usepackage{fixltx2e}
\usepackage{graphicx}
\usepackage{longtable}
\usepackage{float}
\usepackage{wrapfig}
\usepackage{rotating}
\usepackage[normalem]{ulem}
\usepackage{amsmath}
\usepackage{textcomp}
\usepackage{marvosym}
\usepackage{wasysym}
\usepackage{amssymb}
\usepackage{capt-of}
\usepackage{hyperref}
\tolerance=1000
\usepackage{minted}
\usepackage{color}
\usepackage{listings}
\usepackage{grffile}
\usepackage[inline]{enumitem}
\usepackage{setspace}
\usepackage{xcolor}
\hypersetup{
colorlinks,
linkcolor={red!50!black},
citecolor={blue!50!black},
urlcolor={blue!80!black}
}
\usepackage{setspace}%% The linestretch
\singlespacing
\usepackage[format=hang,indention=0cm,singlelinecheck=true,justification=raggedright,labelfont={normalsize,bf},textfont={normalsize}]{caption} %
\usepackage{vmargin}
\setpapersize{A4}
\setmarginsrb{2.5cm}{1cm}% links, oben
{2.5cm}{2cm}% rechts, unten
{12pt}{30pt}% Kopf: Höhe, Abstand
{12pt}{30pt}% Fuß: Höhe, AB
\usepackage{upquote}
%  use straight quotes when printing a command in minted
\AtBeginDocument{%
\def\PYZsq{\textquotesingle}%
}
\definecolor{mintedbackground}{rgb}{0.95,0.95,0.95}
\setlength{\parindent}{0pt}
\setlength{\parskip}{\baselineskip}
\definecolor{mintedbackground}{rgb}{0.95,0.95,0.95}

%% I detest indentation in footnotes etc, so try this:
\makeatletter
\renewcommand\@makefntext[1]{\noindent\makebox[0em][r]{\@makefnmark}\footnotesize#1}
\makeatother
%% the makeatletter and makeatother are required to allow me to
%% to change the macro beginning with an @. (though when I call it
%% I don't use the @ ... 

\renewcommand\scriptsize\normalsize

\author{Martin Jakt\thanks{University of Nordland, Norway}}
\date{\textbf{Bioinformatics \& Genomics}}
\title{\textbf{Perl Assignment 1} (2015-09-11)}
\hypersetup{
 pdfkeywords={},
  pdfsubject={},
  pdfcreator={}}
\begin{document}

\maketitle
%\tableofcontents

\section{The story so far}
\label{sec-1}
In this week's classes you learnt\footnote{I hope anyway} how to: 
\begin{enumerate}
\item write and run a perl script on your computer.
\item access the arguments given to the script when it is run eg. \texttt{\$ARGV[0]}.
\item assign a value to a scalar variable eg. \texttt{\$score=10;}.
\item split a string (piece of text) using \texttt{split //, \$ARGV[0]} and store each character
  as an element of an array (a list).
\item use a \texttt{for} loop to go through each element of an array and do something with it.
\item use the \texttt{eq} operator to check whether two strings (or characters) are the same or different.
\end{enumerate}

From my previous experience of teaching how to write and run scripts this
is perhaps reasonable progress. However, there were some issues that slowed us
down:

\subsection{The location of files and the working directory}
\label{sec-1-1}
One error message that I saw several times when you were trying to
run your scripts was the, "No such file" error. Which suprised you since
you had saved the file, so why couldn't the computer find it. So far we've been running our scripts by explicitly
calling the Perl interpreter to run the script, eg. (\texttt{perl scriptname.pl arg1 arg2}).

When you do that, you just type \texttt{perl}, and the operating system (OS)\footnote{technically this is
the shell, but let's keep it simple.} magically knows to run the perl interpreter.
The perl interpreter (like pretty much all other programs) is just a file that is lying around
somewhere on your disk that happens to have been marked as an executable file (i.e. a file
that can be run). There are 1000s of files lying around on your computer (and many of those could be called 'perl'
or in windows 'perl.exe'), so how
does your operating system decide what to do when you just type perl into the command line?
The answer to this is that the operating system keeps a small list of directories that it
looks in to find programs. This list of directories is stored
in the \texttt{PATH} environment variable. On Macs you can see the value of it by
typing \texttt{echo \$PATH} in the terminal. on Windows you can simply type \texttt{PATH} into your command
window. (In both cases you obviously have to hit the 'Return' (also known as the 'Enter') key.)

So when you type \texttt{perl scriptname.pl arg1 arg2} and hit 'Return' in your terminal window,
the OS looks at the first word (in this case 'perl') of the line, and then goes through each directory in the \texttt{PATH}
variable and searches for a file whose name matches that first word. When it finds a file with
a matching name it runs (executes) that program and gives it the rest of the words you typed
as arguments. In this case the perl interpreter assumes that the first argument passed to it
is the name of the file that contains the script, and that the remaining arguments should be used
by the script \footnote{There is also a way in which we can specify optional arguments to the perl
interpreter which are not passed to the script, but let's ignore that for now}.

Here then is the first problem. How does the perl interpreter find your script? In this case your
script is just a file somewhere on your disk, and note there can be many files with the same name.
Hence we have to specify the location of the script. There are two ways in which this can be done:
\begin{itemize}
  \item Specify the full path (location). On Windows this includes the drive letter (representing a physical
    disk). So for a script called test1.pl in my scripts directory I could write:\\
    \verb|perl C:\Users\lmj\apps\Perl\scripts\test1.pl arg1 arg2|\\
    On Unix systems (eg. Macs) the absolute part starts with a \texttt{/} character representing
    the root of the file system, directories then form a tree like structure from this. Your
    files will be in your home directory which I believe is called \texttt{/Users/user\_name} where
    \texttt{user\_name} is your own user name. Hence on the mac the above would be written like:\\
    \verb|perl /Users/lmj/apps/Perl/scripts/test1.pl arg1 arg2|
  \item Specify the path relative to your current working directory. Eh? What is this 'current working directory'
    thing? Whenever you run a program it is associated with a specific directory within
    the computer's file system (this is called its 'current working directory' and we can simplify and say
    that the program runs in that location). If you start a program from the command line, then it will
    inherit your current working directory. On windows you can see where you are by calling the
    cd (change directory) command without any arguments, i.e. \texttt{cd}. On the Mac (i.e. Unix) you can use the \texttt{pwd} command.

    When using the command line (i.e. the terminal or console) you change your working directory by using the \texttt{cd} command. \texttt{cd ..}
    means go up one level in the directory tree and \texttt{cd dirname} means enter the directory
    called 'dirname' (substitute a real directory name for 'dirname'). The easiest way for you to run your
    scripts then is to use the \texttt{cd} command to change your current working directory to
    the one where your script is (this we normally refer to as 'move to the directory...'). Then you can
    simply run the script without specifying any directory name, eg.:\\
    \texttt{perl test1.pl arg1 arg2}
\end{itemize}
Note that in the above examples \texttt{test1.pl}, \texttt{arg1} and \texttt{arg2} represent the name of your script and whatever arguments your
script takes. \emph{Do not} just copy this blindly.

\subsection{Data, variables and functions}
\label{sec-1-2}
I also noticed when looking at your scripts that some of you were writing something along the lines of:

\begin{minted}[fontsize=\scriptsize,bgcolor=lightgray,linenos]{perl}
#!/usr/bin/perl -w

@ARGV;
print "$ARGV[0] $ARGV[1]";
\end{minted}

\texttt{@ARGV} is an array containing data. It does not specify any action (i.e. it's a noun
not a verb) and writing it on it's own doesn't actually do anything. This is different from
R where simply writing the name of a variable either returns it (from a function) or prints
it to the screen. But that is R, where mysterious things happen automatically without you
knowing quite why or what it is that is happening.\footnote{eg. make a linear model with
R using \texttt{lm}, and look at what happens when you type the name of your model as
compared to the data that the model actually contains.} Perl is simpler than that.

In the above script, nothing really happens until line 4 (well, line 3 might give an
error) where the \texttt{print} function is called. Functions are like verbs, they
make something happen. In this case the data (present in the array \texttt{@ARGV})
is printed to the screen. Every statement (a piece of code ending with a \texttt{;})
has to contain a function of some sort, otherwise nothing will happen (well, maybe
you'll get an error or a warning of some sort). Most functions return a value, and
this can be used either as part of a conditional statement (i.e. if / else)
or be assigned to some variable. For example if we had a function called \texttt{mean}
that calculates the average of an array of values it would usually return the
calculated mean, so that the user (i.e. the programmer) can do something with it:

\begin{minted}[fontsize=\scriptsize,bgcolor=lightgray,linenos]{perl}
## in this case we have an array of values @values
## that has been defined somewhere else in this script
## we call the mean() function on this and it returns the mean
## value. We can use this to assign to a new variable:

$mean = mean(@values);

## or we can make use of it in a conditional statement:

if( mean(@values) > 0 ){
  ## Do something clever here
}else{
  ## Do something different here
}
\end{minted}

One thing that may be a little confusing is that we only sometimes use brackets.
In perl they are often not needed, but they can make it easier to read
the resulting code (even if a bit ugly in certain cases). However,
when you define your own functions you will need to use brackets to make it
clear to the interpreter that you are referring to a function. 

There is a special class of functions called operators. Examples include the assignment
(\texttt{=}), the numerical equality (\texttt{==}), the string equality (\texttt{eq}) and
the less than (\texttt{<}) operators. An incomplete, but much longer list, is given in
the introduction to Perl I sent to you before. The use of these may make it look like
we can have statements without functions, but in fact they are just shorthand ways
of writing out functions.\footnote{There are actually some technical differences that
are important. Most importantly the operators can directly change the values of their
operands.} We refer to the arguments that are passed to operators as their operands
rather than their arguments. eg:

\begin{minted}[fontsize=\scriptsize,bgcolor=lightgray,linenos]{perl}
$a = $b + $c;
\end{minted}
Here we make use of two operators and three operands. \texttt{\$a} and the expression \texttt{\$a + \$b} are
the left and right operands of the \texttt{=} (assignment) operator.
\texttt{\$b} and \texttt{\$c} are also the left and right operands of the \texttt{+} operator
which returns the sum of its operands. The \texttt{=} operator is a bit special since it
does not return a value, but instead changes the value of its left operands (there are
a number of other operators that also do this).

The best way to think about this is that statements are like sentences. If you just string
a bunch of nouns together they do not mean very much or anything. If you just say, 'money'
I can't know whether you want me to give you money, or you want to give me money, or maybe
that you just like money. I might make a guess, and acting on it, get into some degree of
trouble. Generally it's better when the computer (well, the program, or the operating system, which
comes down to the person writing the code) doesn't try to guess too much. Guessing too much
leads to strange and unsafe behaviour.

\subsection{Assignments}
\label{sec-1-3}
All, or most of you, have already written a script that takes two words specified
on the command line and counts how many of the letters in the words match.
You did this by doing something along the lines of:

\begin{minted}[fontsize=\scriptsize,bgcolor=lightgray,linenos]{perl}
#!/usr/bin/perl -w
use warnings;

## Remember, anything following a # on line is a comment
## and is not interpreted. (Except when you use the
## $# notation to get the largest index of an array.)

@s1 = split //, $ARGV[0];
@s2 = split //, $ARGV[1];

$score = 0;
for $i(0..$#s1){
  if($s1[$i] eq $s2[$i]){
    score++; ## this is short for $score = $score + 1
  }
print "the score is: $score \n";
\end{minted}

This will run into difficulty when the first word is longer than
the second one. Try running the script with different length words
and see what happens. Read the error message(s) and try to understand.

\subsubsection{Step 1. Fix this}
\label{sec-1-3-1}
This is perl, so there are many ways to fix this problem. The simplest
one is to change the loop so that it stops before \texttt{\$i} becomes
too large. You could use an additional variable that is set to the
length of the shorter array (\texttt{@s1} or \texttt{@s2}). 
You can do this using something like:

\begin{minted}[fontsize=\scriptsize,bgcolor=lightgray,linenos]{perl}
$maxI = $#s1;
if($#s2 < $#s1){
  $maxI = $#s2;
}
## then use $maxI in the for statement
\end{minted}

You can also make use of the \texttt{break} statement within the loop.
Google perl and break to find out more. This is used to exit a loop
early:

\begin{minted}[fontsize=\scriptsize,bgcolor=lightgray,linenos]{perl}
## exit the loop if $i becomes bigger than 10:
for $i(0..$#s1){
  if($i > 10){  ## if $i is larger than 10
    break;  ## exit the loop.
  }
}
\end{minted}

\subsubsection{Step 2. Get scores for all alignments without internal gaps}
\label{sec-1-3-2}
Previously you've simply compared the letters in the same positions
within the two individual words. That is given the two words \texttt{ATGCA}
and \texttt{ATATG} you have done the comparison like this:
\begin{verbatim}
ATGCA
||***
ATATG
\end{verbatim}
Though it would seem more reasonable to do:
\begin{verbatim}
ATATG
  |||
  ATGCA
\end{verbatim}

For this part calculate alignment scores for all of the possible alignments
that do not contain interal gaps. That is, for:

\begin{minipage}{\linewidth}
\begin{verbatim}
ATATG       ATATG      ATATG     ATATG     ATATG
    *          **        |||      ****     ||***
    ATGCA      ATGCA     ATGCA    ATGCA    ATGCA

 ATATG       ATATG      ATATG     ATATG 
 ****        **|        **        *
ATGCA      ATGCA     ATGCA    ATGCA    
\end{verbatim}
\end{minipage}

Try both with and without scores for terminal gaps. You can
also use a constant gap penalty 
or use separate penalty for gap insertion and extension.

Print out a representation of the best scoring alignment.

\subsubsection{Step 3. An ASCII art dot plot}
\label{sec-1-3-3}
Modify the script from the previous step so that it prints
out a dotplot that visualises the possible ways to align
the sequences. For example, for the two sequences \texttt{ATGGGCGA}
and \texttt{ATGCGATT} print out something like the following:
\begin{spacing}{0.75}
\begin{verbatim}
  ATGGGCGA
A *      *
T  *     
G   *** *
C      * 
G   *** * 
A *      *
T  *      
T  *      
\end{verbatim}
\end{spacing}
Here I have used the \texttt{*} to indicate a match. Maybe a bad choice
as I've used it to indicate mismatches above. This is not the prettiest
thing in the world, but it's a start.

\subsubsection{Step 4. Begin a Needleman-Wunsch}
\label{sec-1-3-4}
The first step of the Needleman-Wunsch algorithm is to fill in a matrix
of scores that are then used to find the optimal alignment. In order to
do this you need to make use of a two-dimensional array to store the
individual position scores in. We haven't yet covered this, but in perl
the values of a two-dimensional array can simply be entered by specifying
two sets of square brackets, as in:

\begin{minted}[fontsize=\scriptsize,bgcolor=lightgray,linenos]{perl}
## to initialise a two dimensional array required for the Needleman-Wunsh
## algorithm
## @s1 and @s2 hold the characters of the words used as defined above..
for $i(0..@s1){
  for $j(0..@s2){
    $scores[$i][$j] = 0;
  }
}
\end{minted}

Remember that our matrix has to have one more row and one more column
than the lengths of the sequences and that in the above code \texttt{@s1}
will evaluate to the length of first word.

Initially use only a simple fixed gap penalty for the calculations. Using
different penalties for gap initiaion and extension is a little bit more
complicated and does run into a problem that is usually omitted from
explanations.

\subsubsection{Optional enhancements}
\label{sec-1-3-5}
If you get all of the above done and wish to proceed further. Then please do so.
The two obvious next steps are:
\begin{enumerate}
\item Record the paths that give rise to the scores in the score matrix. That is
  all the little arrows that we draw in the figures. Again, you can use separate
  two-dimensional arrays to store whether or not a path was taken, or you
  can do something a bit more clever (here bitwise operations seem a natural fit).
\item Use affine gap penalties (i.e. a lower penalty for gap extension than insertion).
  This is not straightforward as you need to look backwards and there sometimes exist
  more than two reasonable options. But you can have a go.
\end{enumerate}

You can also consider to read the sequences from one or two files instead of specifying
them on the command line. This is all explained in the Introduction to Perl that I made
available to you before (including all the code needed to read in sequences from a multi
sequence fasta file). See the reading and writing data and regular expression sections as
well as the last somewhat useful example part. Remember to read the comment sections within
the code examples as these explain a lot of what is done.

You can also have a look at the dotplot\_svg.pl script that I uploaded before and see if it
makes more sense to you now. If it does, then see if you can work out how to use
the SVG module to output somewhat more pleasant output.

The last step of the Needleman-Wunsch is to backtrack to find the optimal path(s). This
is simple when there is only a single path, but it is more difficult to find all optimal
paths. It is most commonly performed using recursion\footnote{a function 
that calls itself}, and this introduces some additional problems. However, you don't have
to use recursion and I would suggest you have a bit of a think about it and see if you 
can come up with a good idea.


\end{document}



