% Created 2015-05-28 Thu 13:42
\documentclass[11pt]{article}
\usepackage[utf8]{inputenc}
\usepackage[T1]{fontenc}
\usepackage{fixltx2e}
\usepackage{graphicx}
\usepackage{longtable}
\usepackage{float}
\usepackage{wrapfig}
\usepackage{rotating}
\usepackage[normalem]{ulem}
\usepackage{amsmath}
\usepackage{textcomp}
\usepackage{marvosym}
\usepackage{wasysym}
\usepackage{amssymb}
\usepackage{capt-of}
\usepackage{hyperref}
\tolerance=1000
\usepackage{minted}
\usepackage{color}
\usepackage{listings}
\usepackage{grffile}
\usepackage[inline]{enumitem}
\usepackage{xcolor}
\hypersetup{
colorlinks,
linkcolor={red!50!black},
citecolor={blue!50!black},
urlcolor={blue!80!black}
}
\usepackage{setspace}%% The linestretch
\singlespacing
\usepackage[format=hang,indention=0cm,singlelinecheck=true,justification=raggedright,labelfont={normalsize,bf},textfont={normalsize}]{caption} %
\usepackage{vmargin}
\setpapersize{A4}
\setmarginsrb{2.5cm}{1cm}% links, oben
{2.5cm}{2cm}% rechts, unten
{12pt}{30pt}% Kopf: Höhe, Abstand
{12pt}{30pt}% Fuß: Höhe, AB
\usepackage{upquote}
%  use straight quotes when printing a command in minted
\AtBeginDocument{%
\def\PYZsq{\textquotesingle}%
}
\definecolor{mintedbackground}{rgb}{0.95,0.95,0.95}
\setlength{\parindent}{0pt}
\setlength{\parskip}{\baselineskip}
\definecolor{mintedbackground}{rgb}{0.95,0.95,0.95}
\author{Martin Jakt\thanks{University of Nordland, Norway}}
\date{\textbf{Bioinformatics \& Genomics}}
\title{\textbf{Introduction to Perl} (2015-06-02)}
\hypersetup{
 pdfkeywords={},
  pdfsubject={},
  pdfcreator={}}
\begin{document}

\maketitle
%\tableofcontents

\section{Using Perl}
\label{sec-5}

Perl is a useful programming language whose principles can be learnt
within a short period of time allowing researchers not familiar with
programming to quickly become able to automate a variety of processes.
Although not an official acronym, Perl is often referred to as standing
for, 'Practical Extraction and Reporting Language'; and this is pretty much
what Perl makes easy.

Perl has been used extensively within the field of Bioinformatics (see
Bioperl, \url{http://www.bioperl.org}) though recently it has been overshadowed to
some extent by the use of R for statistical analyses of data. However,
Perl remains widely used though, and many tools used for the analysis of next-generation sequencing
have been written in Perl. R is incredibly useful when you have
regular data structures that can be expressed as arrays or matrices;
however it is unsuitable for describing irregular types of data (eg.
structures of genes, etc.) where it may be necessary to iterate through
the elements of a data set. Compared to R, Perl is a much more general
programming language that can be applied to a much wider set of
problems.

The motto of Perl is, 'There is more than one way to do it'. And in Perl
this is very true; the same logic can be expressed in a number of
different ways and masters of Perl will sometimes delight in their
ability to fit a very large amount of functionality into a small amount
of code. This is kind of neat, but can lead to code that is difficult to
understand and should not be encouraged for code that will
actually be used. The flexibility of Perl also means that it can be
difficult to read other people's code as they may use a very different
style of coding to ones own. Perl can also be quite a dangerous language
and it is often said that it gives the user more than enough rope to
hang themselves with.

\subsubsection{Variables in Perl}
\label{sec-5-0-1}

In order to handle information within a program we assign values to
variables and then manipulate these according to the flow of the
program. Perl provides three different types of variables:

\begin{itemize}
\item Scalar variables: these take a single value (usually a number or some text) 
and are denoted by a \texttt{\$} prefix, eg. \texttt{\$var}.

\item Arrays: these contain an ordered series of values that are accessed by their
position. Arrays are denoted by an \texttt{@} prefix, eg. \texttt{@array}.
Individual values are accessed as scalars, using square brackets to
indicate the position, eg. \texttt{\$array[3]} accesses the fourth element of
\texttt{@array} (the fourth rather than the third as we count from 0).

\item Hashes (or associative arrays): these hold key-value pairs and are
denoted by the \texttt{\%} prefix, eg. \texttt{\%hash}. Individual elements are again
accessed as scalars, but this time using curly brackets, eg.
\texttt{\$hash\{key\}}. The key value can be anything that can be assigned to a
scalar (numbers, text, and references).
\end{itemize}

\subsubsection{Assigning variables}
\label{sec-5-0-2}

The values of variables can be assigned directly in the program's source
code, but are more frequently assigned through the command line
arguments (see below) or by the program reading input (data or
configuration) files (see lower section). Scalars are the simplest:

\begin{minted}[fontsize=\scriptsize,bgcolor=lightgray,linenos]{perl}
$var1='hello'; 
$var2="world";
$var3=3.14;
\end{minted}

Strings (i.e. text elements) can be assigned using either single \texttt{’} or
double " quotes. The use of double quotes expands variables within the
quoted text such that:

\begin{minted}[fontsize=\scriptsize,bgcolor=lightgray,linenos]{perl}
$var4="goodbye $var1";
\end{minted}

will assign the text "goodbye world" to the variable \texttt{\$var4}.
In contrast:

\begin{minted}[fontsize=\scriptsize,bgcolor=lightgray,linenos]{perl}
$var4='goodbye $var1';
\end{minted}

will assign the text 'goodbye \$var1' to \texttt{\$var4} (without the quotation
marks!).
Double quotes also allow escape codes such as \texttt{\textbackslash{}n \textbackslash{}t} to be interpreted
as newline and tab characters respectively.

Arrays can be assigned in a number of ways, occassionally directly in
the code:

\begin{minted}[fontsize=\scriptsize,bgcolor=lightgray,linenos]{perl}
@ar1 = (1, 2, "three");
\end{minted}

An empty array can also be created and then extended by adding elements.
This can be done by either using the \texttt{push} function or by using
subscripts beyond the range of the array:

\begin{minted}[fontsize=\scriptsize,bgcolor=lightgray,linenos]{perl}
## text following a # character are treated as comments

@ar1 = (); ## creates an empty array of length 0 
push @ar1, "hello"; ##extends this array to have a length of 1

$ar1[2] = "three"; 
## the array now has a length of three, but an undefined value in the second position 
## $ar1[1]
\end{minted}

In most cases, elements of an array will be assigned to values found in
input files containing the data to be analysed, rather than being
defined directly in the code as above.

Hashes (associative arrays) that store key value pairs are defined in a
similar way to arrays. Again the actual values are usually obtained from
input files, but can also be defined in the code.

\begin{minted}[fontsize=\scriptsize,bgcolor=lightgray,linenos]{perl}
%kv1 = ();
## this creates an empty hash structure. It is actually not necessary to
## declare it, but one can directly assign elements of the hash:
$kv1{1} = "one";
$kv1{2} = "two";
$kv1{'three'} = 3;

## this hash could also have been created in a single line :
%kv1 = (1 => "one", 2 => "two", 'three' => 3);

## to access the elements of an associative array we obtain
## the keys of the hash using the keys command.

@keys = keys %kv1;
## print the first value associated with the first key:
print "$keys[0] $kv1{$keys[0]}\n";

## the \n simply defines a newline character
\end{minted}


Scalars, arrays and associative arrays can be combined to create
arbitrarily complex data structures. Hence you can have hashes of arrays
and arrays of hashes and so on. To fully use more complicated data
structures requires an understanding of the reference. A reference is a
value that points to another piece of data by providing the memory
address of that data. For example, an array of hashes is encoded as an
array of references to hashes. To obtain the value of data referred to
by a reference the reference must be dereferenced. Perl has
a number of different ways in which this can be done, but these will not
be explained in depth here as it can get a bit messy. 

Semicolons: you may have noticed that in the above examples almost every
line ends with a semicolon. In Perl (and in many other languages), the
semicolon is used to denote the end of statements. This means
that single statements can be spread across several lines and that a
single line can contain a number of statements. This can greatly aid the
readability of the code.

\subsubsection{Data types}
\label{sec-5-0-3}

In the above examples we assigned values to variables without caring
about what kind of data we used. For example consider the following:

\begin{minted}[fontsize=\scriptsize,bgcolor=lightgray,linenos]{perl}
$var1 = "one";
$var2 = 2;
$var3 = $var1 + $var2;  ## !!??!!
\end{minted}

Here we have assigned the value of \texttt{\$var1} to a piece of text (which we
will refer to as a string from here on) whereas \texttt{\$var2} has been
assigned a numeric value. Perl is a dynamically typed language; that
means that you do not have to explicitly define what type of value a
variable contains. This is convenient when writing a script (essentially
a small program), but this does make it easier to make mistakes in more
complicated situations. In the above example, the third line doesn't
make sense, and will generate an error. In this case it is obvious from
the code, but in most real world situations the values will be read in
from an external file produced by some other program or person in which
case finding the reason for the problem may not be so simple.

Perl essentially has three data types, strings, numeric values and
references. References are necessary for making more complex data
structures and to allow variable values to be modified by functions. As
mentioned above though, references will not be covered in much depth in
order to keep things reasonably simple. The string and
numerical data types are fairly straightforward, though there are a few
potential problems (common to essentially all computer programming):

\begin{itemize}
\item Numeric values do not have infinite precision. For example (1/3) is
not equal to (0.1/0.3).

\item Numeric values can not be arbitrarily large. On my machine the
maximum value Perl can handle is somewhere between 1e308 and
1e309. That's a pretty large number which you might not think that
you will ever need.  However, it is smaller than the factorial of
171, and this is something you may run across in statistical
equations.

\item Mathematical operations can result in illegal numbers, eg. 1/0. If
your program carries out any calcuations you need to be aware of
this and how Perl handles the resulting values.

\item Text is actually not that simple. From the beginning, the end of
lines has been encoded differently in Windows (i.e. DOS), MacOS and
Unix. In Unix an end of line is encoded with a newline character, on
Windows, a newline character followed by a carriage return, and on
MacOS it might be just a carriage return (to be honest I
forget). This can cause trouble as text files are usually written
and read line by line (i.e.  new lines indicate a new section of
data). The simplest way to avoid trouble is simply never to touch
Macs or Windows machines, but that can be difficult at times.

\item These days text encoding is rather complicated, as it has been
expanded to cater to a range of languages and character sets
(eg. Arabic, Chinese, Japanese, Thai, etc..). This is not
straightforward and several conflicting encodings have been
developed. For bioinformatics you usually do not have to care; but
you have to be aware of potential problems when handling text that
contains unstructured descriptive data. Such text may contain
names, or places written in glyphs that require Unicode
encoding. Such descriptions may even contain characters that look
like normal roman letters, but which have been encoded differently.
Google, 'halfwidth fullwidth characters' to confuse yourself.

\item Sorting. Numbers and strings are obviously sorted
differently. Consider that \texttt{(12 > 8)}, but \texttt{('12' < '8')}. In the latter
case we are comparing strings through a lexicographic comparison
where the first character is the most significant for the
sort. Since 8 is larger than 1, "8" is also larger than "12". In
Perl sorting is lexicographic by default, and a numeric sort has to
be explicitly specified. This is sometimes problematic when a mix of
numerical and character based identifiers are used and the reason
that you often see the following chromosome ordering:
1,10,11,12,\ldots{},19,20,21,3,4,5,\ldots{},9,X,Y.
\end{itemize}

\subsubsection{Program flow: loops and conditionals}
\label{sec-5-0-4}

We use computer programs to automate repeated processes; that is to
carry out the same or similar operations on a large number of data
points. This is (usually) done by iterating over a collection of data
until some condition is met. That condition is often simply that we have
no more pieces of data to look at, but the condition can also be that a
solution to some problem has been found, or anything that you can think
of. This process is referred to as looping.

Similarly programs need to be able to handle the data differently
depending on what it is. This is handled by conditional statements.
Conditional statements are also used in lots of other cases including to
control loops. Consider the following statement that checks for the
equality of two variables.

\begin{minted}[fontsize=\scriptsize,bgcolor=lightgray,linenos]{perl}
## $a and $b are two variables whose values are specified somewhere else in the program.
if($a == $b){
  ## then do something. For example increase the value of $b
  $b = $b + 1;
}
\end{minted}

There are a few things to mention here. The first is the use of the \texttt{==}
operator. This tests for numerical equality. It is very important not to
confuse this with the \texttt{=} operator which assigns values. Comparison
operators can be thought of as returning a TRUE or a FALSE value. If a
TRUE value is obtained then the conditional statement is carried out,
and if FALSE not. Perl doesn't actually have explicit TRUE and FALSE
values, but any non-0 value is considered as TRUE and a value of 0 is
considered as FALSE. To confuse things the use of the assignment
operator returns the value that was assigned and this can cause some
rather specific problems. Consider:

\begin{minted}[fontsize=\scriptsize,bgcolor=lightgray,linenos]{perl}
$a = ($b = 10);
## $a is now assigned to the value of 10

## this conditional statement will always evaluate to TRUE
if( $a = 25 ){
  ## this will always be executed
}

## but this will never evaluate to TRUE
if($a = 0){
  ## this part of the program will never be reached
}
\end{minted}

The second thing to mention is the use of the curly brackets (\{and\}). In
Perl (and quite a few other programming languages) these are used to
break the code up into blocks of code that can be conditionally executed
(or looped over, which is kind of conditional). In Perl, blocks of code
can have their own scope by using the \texttt{my} keyword. This means that a
variable which is defined within a block of code is not visible outside
of that block of code. This is very useful for more complicated programs
where it is easy to accidentally use the same variable names to represent
different properties.
Consider the following snippet:

\begin{minted}[fontsize=\scriptsize,bgcolor=lightgray,linenos]{perl}
## We start in the global scope. Variables defined here will be visible and modifiable
## anywhere within the main body of the code (though not in external functions).

$a = 10;
{
  $a = 20;
}

print "a is $a \n";
## will print 20. However if we do:

{
  my $a = 30;
  ## $a will be equal to 30 only within this block of code
}

print "a is now $a \n";
## does not print 30, as we $a was declared using the
## my keyword.
\end{minted}

It is good practice to use \texttt{my} and the related \texttt{our} keyword throughout
the code as it will make it easier to catch a range of different types
of errors. This can be enforced by \texttt{use strict;}. Google for more!

Looping can be used if, for example you have an array of values that you wish to
obtain the mean value of. To do this we wish to find the sum of the
values and divide by the length of the array. As always in Perl there
are a number of ways in which this can be done:

\begin{minted}[fontsize=\scriptsize,bgcolor=lightgray,linenos]{perl}
## @ar is an array of values specified somewhere else in the program.
## ++ is an increment operator that increases the value of its operand
## by one each time it is called.
## += is an increment operator that increases the value of its left operand
## by the value of its right operand.

## to loop through the values we can use a classic for loop:
$sum = 0;
for( $i=0; $i < @ar; $i++){
  $sum += $ar[$i];
}

$mean = $sum  @ar;
## when an array variable is used in an expression it can can evaluate to either the array itself
## or to a scalar value equal to its length. When it's not clear as to whether the scalar or array
## value is indicated, the scalar value can be enforced by the scalar function.

## We can also use a range specified loop and make use of the fact that in Perl
## $#ar will evaluate to the higest index of an array (i.e. the length minus one)

for $i(0..$#ar){
  $sum += $ar[$i];
}

## we can also use a similar expression;
for $v(@ar){
  $sum += $v;
}

## alternatively we can use a while loop by specifying the index variable outside
## of the loop statement;
$i = 0;
while($i < @ar){
  $sum += $ar[$i];
  $i++;
}
\end{minted}


These are not the only ways in which you can loop through values or data
structures, but they probably represent the most common usages.

\subsubsection{Reading and writing data}
\label{sec-5-0-5}

To read or write from a file we use a filehandle. This is just an
identifier associated with the file and the reading or writing process.
To write to a file we usually use the \texttt{print} function. Using \texttt{print}
without specifying a filehandle will lead to the text being printed to
STDOUT. In most cases this means your terminal screen, but STDOUT can
also be piped to other processes as demonstrated previously in this
guide. To open a text file and read a line at a time:

\begin{minted}[fontsize=\scriptsize,bgcolor=lightgray,linenos]{perl}
## we wish to read from a file specified by the varialbe $fname

open(IN, $fname) || die "unable to open $fname $!\n";
## here IN becomes specified as the filehandle (This is one of the few cases
## where we use an undecorated string literal as an identifier).
## The second half of the statement uses the '||' operator which simply means 'or'.
## If we are unable to open the file then the program will print out the warning statement
## following die and exit. $! is a magic variable that contains the error string.

## to read all of the lines we can make use of a while loop
while(<IN>){
  ## this will assign the text of each line to another magical variable, $_
  ## we can print this out to STDOUT by calling
  print;   ## without arguments this prints $_ to STDOUT

  ## normally we would do something useful first by processing the data in the line.
  ## but more of that later.
}
\end{minted}



To write to a file we also use open, but modify the filename to indicate
that we wish to write to a new file by prefixing the name with a '>'
character. If a file of the same name exists it will be overwritten. If
we wish to append to an existing file we can use '>>'.

\begin{minted}[fontsize=\scriptsize,bgcolor=lightgray,linenos]{perl}
## given that we wish to write something to a file specified by the
## $fname variable.
open(OUT, ">$fname") || die "unable to open $fname $!\n";
## write out the multiplication table (1..10) to the file
## first write out some column headers
for $i(1..10)\{
  print OUT "\t$i";
}
print OUT "\n";

for $i(1..10){
  print OUT $i;
  for $j(1..10){
    print OUT "\t", $i * $j;
  }
  print OUT "\n";
}

close OUT;
\end{minted}

\subsubsection{Regular Expressions}
\label{sec-5-0-6}

Regular expressions are widely used to describe text patterns. The Perl
implementation of regular expressions is perhaps one of the best and
most powerful ones available and a large part of the power of Perl comes
through its ability to make use of regular expressions.

As mentioned previously regular expressions are used to identify matches
to generalised text patterns in strings. There are a very large number
of tutorials on how to use regular expressions in Perl available on the
net and we will only provide a very short introduction here.

In Perl, regular expression matching makes use of the \texttt{=\textasciitilde{}} operator,
where the left operand contains the text to be searched for matches to the
pattern given by the right operand. Some examples:

\begin{minted}[fontsize=\scriptsize,bgcolor=lightgray,linenos]{perl}
## The left operand is usually a variable, but for clarity we'll use
## plain strings.

## The regular expression is usually written as follows:
## "some string to be tested" = m/ a regular expression /
##
## the character immediately following the m delimits the regular expression. If you wish to
## include this character within the regular expression it will need to be escaped by placing
## a \ in front of it. For regular pattern matching you do not need to specify the
## m if you are using the forward slash as the delimiter. This is the most common way to write it.
## So to check if an expression looks like the name of a Hox gene we can do:

"HoxA3" =~ /hox[a-z][0-9]+/;

## Normal characters are matched directly, characters within square brackets [] represent a character
## class (any character specified will allow a match). In the above example, the regular expression
## will fail to recognise the left operand since the regular expression is case sensitive. To overcome
## this we can do:

"HoxA3" =~ /hox[a-z][0-9]+/i;

## we could also specify a character class at each position, but this would be ugly:
"HoxA3" =~ /[hH][oO][xX][A-z][0-9]+/;

## which reads as: h OR H followed by o OR O followed by x OR X 
## followed by a single character between A and z
## followed by at least one number. But that is pretty ugly.

## if you wish to use a different delimiter, like the # character you can write it like:
"HoxA3" =~ m#hox[a-z][0-9]+#i

## this can be useful when trying to match directory names that contain lots of forward slashes.

## The above expressions on their own do nothing as we do not make use of the returned value
## To actually use a regular expression we make use of conditionals, eg...

if("HoxA3" =~ /hox[a-z][0-9]+/i){
  ## we have Hox gene, do something here..
}
## to substitute words we can use the s modifier. We may wish to substitute spaces within a
## a string with underscores.
$string = "Goodbye cruel World";
$string =~ s/ /_/g;

## here we also make use of the g (global) modifier to replace all instances rather than just the first
## match.
\end{minted}

Regular expressions make use of a number of special characters and
modifiers to represent textual patterns. The characters represent
character classes, followed by a modifier specifying how many matches
should be present to give a match. In Perl, the most widely used special
characters are:

\begin{itemize}
\item \texttt{.} The dot. This matches any character.

\item \texttt{\textbackslash{}d} A numeric character. Equivalent to specifying [0-9].

\item \texttt{\textbackslash{}s} A space.

\item \texttt{\textbackslash{}S} Non space characters.

\item \texttt{\textbackslash{}w} Word characters (alpha numeric and some others).

\item \texttt{\textbackslash{}b} Word boundaries (tabs, spaces, newlines, punctuation).

\item \texttt{\textbackslash{}t} Tab characters.
\end{itemize}

A character may be followed by a modifier specifying how many times the
character should be present in the text.

\begin{itemize}
\item \texttt{+} 1 or more.

\item \texttt{*} 0 or more.

\item \texttt{?} 0 or 1.

\item \texttt{\{N\}} Exactly N times.

\item \texttt{\{n..N\}} n to N times.
\end{itemize}

Other modifiers can be used to specify where a match should be present:
\texttt{\textasciicircum{}} and \texttt{\$} specify the beginning and end of lines respectively. Note
that \texttt{\textasciicircum{}} inside a character class indicates an inverted character class
(matches characters not present in the class).

Regular expressions can also be used to capture specific subsections of
text. A very common example would be to extract a sequence identifier
from a fasta file. This can easily be done in Perl.

\begin{minted}[fontsize=\scriptsize,bgcolor=lightgray,linenos]{perl}
## $line contains a line from a file. Identifiers begin with the > character.
if( $line =~ /^>(\S+)/ ){
    $seqId = $1;
}
## if brackets are used in the regular expression, the values matching within the brackets
## will be assigned to variables $1 - $9. (Ordered from left to right). If you wish to match
## brackets you will need to escape them with backslashes.
\end{minted}

There's a lot more to regular expressions than this, but this may be enough to get
started with.

\subsubsection{Various operators}
\label{sec-5-0-7}

Operators are symbols that denote specific operations; like regular
expression matching or regular mathematical operations. We have already
come across a few of these, but there are more (and the following list
is not complete).

\begin{itemize}
\item \texttt{+} The addition operator. Returns the sum of the left and right
operand.

\item \texttt{-} The subtraction operator.

\item \texttt{++} The auto-increment operator. Increases the value of its single
operand by 1. There are in fact two different increment operators;
post-increment \texttt{\$v++} and pre-increment \texttt{++\$v}. The former increments
the value after other operations, the latter before. Consider the
difference between \texttt{\$i=5; print \$i++;} and \texttt{\$i=5; print ++\$i;}.

\item \texttt{–} The auto-decrement operator. Opposite of auto-increment.

\item \texttt{+=} The increment operator. Increases the value of its left operand
by the value of its right operand.

\item \texttt{-=} The decrement operator. Opposite of the increment operator.

\item \texttt{*} Multiplication.

\item \texttt{/} Division.

\item \texttt{*=} Sets the value of its left operand to the product of the left
and right operands. Identical to \texttt{\$left = \$left * \$right}.

\item \texttt{/=} As above but for division.

\item \texttt{**} Exponentiation. Returns the value of the left operand to the
power of the right operand.

\item \texttt{.} String concatenation. Concatenates left and right operands.

\item \texttt{.=} Concatenates right operand to left operand.

\item \texttt{==} Numerical equality operator. Returns TRUE if the value of the
left and right operands are equal. Causes an error if either
operand is not numerical.

\item \texttt{!=} Numerical inequality operator. Returns TRUE if the value of the
left and right operands are not equal. Causes an error if either
operand is not numerical.

\item \texttt{eq} String equality operator. Returns TRUE if the strings specified
by the left and the right hand operators are the same.

\item \texttt{ne} String inequality operator. Returns TRUE if the strings specified
by left and right hand operators are not the same.
\item \texttt{>} Numerical greater than. Returns true if left operator is larger than
the right operator.

\item \texttt{<} Numerical less than. Opposite of above.

\item \texttt{>=} Numerical greater than or equal to.

\item \texttt{\%} The modulus operator. Returns the remainder of the division of
  its left operand by its right operand. A very useful operator!
\end{itemize}

This is an incomplete list, but is sufficient to do rather a lot with. Note
that some operators should be used with numerical values and others with strings
(pieces of text). Using the wrong data types will sometimes raise errors, but
can also result in the program silently doing something unexpected (which is the
worst kind of behaviour as it can result in corrupt output).

\subsubsection{A somewhat useful example}
\label{sec-5-0-8}

As an example of something potentially useful we can write a short script
that reads in sequences from a fasta file and identifies sequences that
contain a specific pattern within the first N bases. To do this we'll
make use of most of the techniques outlined above, but we'll also need
to be able to work out options specified by the user on the command
line. The arguments specified to a Perl script are assigned to a special
array called \texttt{@ARGV}, and we'll make use of this array to work out what
the user wants to do.

The following segment contains a full script that you should be able to
run, using the ./scriptname invocation.

\begin{minted}[fontsize=\scriptsize,bgcolor=lightgray,linenos]{perl}
#!/usr/bin/perl -w

## the first line is not really a comment, but is used to make the shell 
## invoke the perl interpreter on the script.

## first check the command line arguments to make sure that the user has specified three arguments.
## the first argument should give the name of the fasta file containing the sequences to be searched,
## the second argument the pattern to look for, and the third argument the maximum distance from the
## beginning of the sequence.

if(@ARGV != 3)\{
  die "usage: script_name fasta_file pattern max_distance_from_edge \n";
}

## we could also use regular expressions to check if the arguments are of the correct type

$seqId = "";
$seq = "";

## open the fasta file and read line by line.
open(IN, $ARGV[0]) || die "unable to open $ARGV[0] $!\n";
while(<IN>){
  chomp; ## this removes the end of line character from $_
  ## does the line look like it contains a sequence identifier?
  if( $_ =~ /^>(\S+)/ ){
    $seqId = $1;
    next;  ## go to the next iteration of the loop
  }
  ## if we have defined a sequence identifer, we will just assume
  ## that the rest of the text contains sequence.
  if(length($seqId)){
    $seq{$seqId} .= $_;
  }
}

## We should now have read all of the sequences into an associative array 
## where the keys are the sequence identifiers. We now go through the 
## sequences and check for the pattern. The identifiers of sequences which 
## match are printed out to STDOUT. 
## We could also print the matching sequences if we wished.

for $seqId(keys %seq){
  if( $seq{$seqId} =~ /^.{0,$ARGV[2]}$ARGV[1]/ ){
    print "$seqID\n";
  }
}

## end of the script!
\end{minted}

This script probably has a few bugs in it. Working out where those bugs
are is a pretty good exercise for honing your Perl skills. Note also
that bad command line arguments can cause all sorts of problems as the
script does not check the arguments given. The script is quite useful
though, as you can use it as a sort of configurable grep to learn more
about regular expressions in Perl.

Be aware that this is not a very memory efficient way of solving the
problem as all of the sequences are read into memory before any
processing is done. This is not only memory intensive, but it's also
slower. It's been written this way to show the use of hashes and to keep
it reasonably short. I've also avoided using custom functions as I've
not included anything about how to write your own functions (subroutines
in Perl). How to write your own functions is probably the first thing
you should look at after this introduction if you wish to start using
Perl seriously.

Good luck with Perl!

\end{document}



