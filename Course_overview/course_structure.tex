% Created 2015-07-16 to. 11:51
\documentclass{scrartcl}
\usepackage[utf8]{inputenc}
\usepackage[T1]{fontenc}
\usepackage{fixltx2e}
\usepackage{graphicx}
\usepackage{longtable}
\usepackage{float}
\usepackage{wrapfig}
\usepackage{soul}
\usepackage{textcomp}
\usepackage{marvosym}
\usepackage{wasysym}
\usepackage{latexsym}
\usepackage{amssymb}
\usepackage{hyperref}
\tolerance=1000
\providecommand{\alert}[1]{\textbf{#1}}

\title{Topics to be covered by B1229F 2015 (Genomikk og Bioinformatikk)}
\author{Lars Martin Jakt\thanks{lmj@uin.no}}
\date{\today}
\hypersetup{
  pdfkeywords={},
  pdfsubject={},
  pdfcreator={Emacs Org-mode version 7.9.3f}}

\begin{document}

\maketitle

\setcounter{tocdepth}{3}
\tableofcontents
\vspace*{1cm}




\section{General Introduction}
\label{sec-1}

\begin{itemize}
\item What do we mean by bioinformatics
\item Why do we need it
\item Some expansive examples of usage.
\end{itemize}
\section{Basic molecular biology.}
\label{sec-2}
\subsection{Central dogma}
\label{sec-2-1}

DNA -> RNA -> protein\\
With some exceptions:
\begin{itemize}
\item RNA viruses
\item Retrotransposons
\item Transposons
\end{itemize}
Transcription is a dynamically regulated process.

Translation efficiency is variable and an (the most?)
important variable affecting protein abundance.

Protein degradation is also dynamically regulated (both by cell
state and through reading of external signals).
\subsection{Gene structure, prokaryotes \& eukaryotes}
\label{sec-2-2}

\begin{itemize}
\item Prokaryotes, Operon, as example perhaps the lac operon.
\item Eukaryotes, promoter, introns, exons, poly A signal.
\item locus control region (eg. B-globin).
\end{itemize}
\subsection{Encoding of amino acid sequence in DNA}
\label{sec-2-3}

\begin{itemize}
\item 4 bases encode 20 amino acids
\item Number of encodings possible with different message length
\item Binary representation ?
\end{itemize}
\subsection{Basic regulation of transcription}
\label{sec-2-4}

\begin{itemize}
\item Transcription factors \& cis acting regions
\item Stabilisation or implementation through changes in chromatin
\item DNA methylation ?
\end{itemize}
\subsection{Information flow in cellular systems (input --> output)}
\label{sec-2-5}

\begin{itemize}
\item Conversion of external signals to cellular decisions
\item How DNA sequence is read: the fundamental requirement to
  be able to identify different sequences
\begin{itemize}
\item Transcription factors and sequence specific binding
\item RNA as a targetting factor (but by design must be self-targetting)
\end{itemize}
\item Stabilisation of state through chromatin and DNA modification
\begin{itemize}
\item Polycomb / Trithorax
\item Histone modifications
\item DNA methylation
\end{itemize}
\item Cell state modifies the response to the cellular environment.
\end{itemize}
\subsection{Genome / Transcriptome / Proteome}
\label{sec-2-6}

What we mean by these terms.
\begin{itemize}
\item Genome: The full set of DNA sequence, a constant.
\item Transcriptome: The full set of RNA molecules present in a cell.
  A function of the genome, the state of the cell, its past activity
  (history), and it's environment.
\item Proteome: The full set of protein molecules present in a cell.
  A time delayed function of the cell's transcriptome and state
  (consider regulated protein degradation).
\end{itemize}
\section{Informatics}
\label{sec-3}
\subsection{Theoretical}
\label{sec-3-1}
\subsubsection{Introduction (what do we mean by Informatics)}
\label{sec-3-1-1}
\subsubsection{What is information (Akaike's information criteria?)}
\label{sec-3-1-2}

Pictures:
\begin{itemize}
\item polar bear in a snow storm
\item tropical fish?, repeated structures.
\item Fourier transforms
\end{itemize}
Text:
\begin{itemize}
\item Redundancy
\item Nonsense
\item Political writing
\end{itemize}
\subsubsection{Linguistic redundancy \& lack of meaning}
\label{sec-3-1-3}

\begin{itemize}
\item I am here now
\item We have previously done
\item Examples from thesis work
\item Job postings (This position requires a good candidate)
\end{itemize}
\subsubsection{Data vs information (the reverse of AIK)}
\label{sec-3-1-4}

\begin{itemize}
\item Picture of a .CEL file (microarray data)
\item Linkage of values to identities to provide information
\item RNA seq data, linked to genome
\item Functional annotation
\item Inferring biological mechanisms\\
DNA methylation in cancer \& histone modification, polycomb proteins ? (too difficult perhaps)\\
  Promoter classes (general classification problems)
\end{itemize}
\subsubsection{Entropy (probably not)}
\label{sec-3-1-5}
\subsection{Practical}
\label{sec-3-2}
\subsubsection{Computers networks \& file systems}
\label{sec-3-2-1}
\subsubsection{Operating systems and applications}
\label{sec-3-2-2}
\subsubsection{Shells (graphical \& command line)}
\label{sec-3-2-3}
\subsection{Mathematical considerations}
\label{sec-3-3}

This is probably best handled using practical classes, maybe try to
combine a morning session of theoretical, followed by a couple of
hours of playing around with R looking at numbers.
\subsubsection{How to look at lots of numbers}
\label{sec-3-3-1}

\begin{itemize}
\item Summarising numbers (eg, mean, median, mode), and their variance:
  Variance, standard deviation, MAD (median of something around the median).
  Use some simple example: how tall is a Norwegian person? How much beer
  does a Norwegian drink?\\
  This has obvious potential be bimodal (men / women), and then there
  is also the impact of age. These are factors of variance, but can also
  be used to consider correlation (for example, plot height vs. beer
  consumption; drinking beer makes you taller!). (BMJ article on beer
  consumption vs heart disease?)
\item Derivatives in general for summarising numbers. Make up some potential
  examples from an experimental condition. (Easy example is the t-statistic,
  we can derive it through a reasoning. We can also consider the correlation
  coefficient).
\item Doesn't always work: Various ways of visualising numbers (theoretical component).
\item Low level plot commands
\item How to visualise distributions from numbers
\end{itemize}
\subsubsection{Distributions}
\label{sec-3-3-2}

\begin{itemize}
\item Cover different types in lecture and how they can arise.
\item Use simulations to derive distributions in practical class.
  eg. taking arithmetic or geometric means, deriving extreme,
  binomial, hypergeometric and other distributions by sampling.
\end{itemize}
\subsubsection{Deriving statistics from distributions}
\label{sec-3-3-3}

\begin{itemize}
\item Explain a couple of commonly used statistical tests, like
  t and F (i.e. Anova)
\item Use sampling of random and non random numbers to derive t
  and F statistics.
\item Look at the cumulative distributions to see the likelihood of
  observing a particular value.
\item Compare observations with values from functions like pf, phyper,
  pt (?).
\end{itemize}
\section{Bioinformatics}
\label{sec-4}

\begin{itemize}
\item Applications of informatics to biology.
\item Defined by the problem domain.
\item Any problem that is difficult approach without a computer.
\item Driven by:
\begin{itemize}
\item Experimental technologies that increase amount of data.
\item Shift towards investigating systems of interactions.
\end{itemize}
\end{itemize}

Bioinformatics can be though of as an enabling technology, similar those
that allow us to produce reagents (eg. plasmids, antibodies, probes), make mutants,
obtain large data sets and whatever you can think of. However, it is different in one fundamental aspect:
\begin{itemize}
\item Bioinformatics is used to analyse your data and (often) provides the final answer
\item How it is used defines the (biological) question asked
\item The output is used to make conclusions which may not be confirmed by other means
\end{itemize}

\textbf{Therefore you really should have some understanding of what the tools you use do}
\subsection{Sequence alignment}
\label{sec-4-1}
\subsubsection{Theoretical}
\label{sec-4-1-1}

\begin{itemize}
\item Scoring alignments
\end{itemize}
Explanation of substitution matrice
\begin{itemize}
\item Nucleotide sequences
\item Insertion, gap and extension penalties
\item Amino acid sequences
\item Development of scoring / substitution matrices
\item Global vs. local
\item Algorithms
\item Visualising alignments: eg. dotplot
\item Needleman-Wunch / Smith-Waterman (but what to say about these)
\end{itemize}
\subsubsection{Practical}
\label{sec-4-1-2}

\begin{itemize}
\item Manual alignment (or just scoring) exercise ?
\item Gapped
\item Gapless
\item Global alignment
\item Local alignment
\end{itemize}
\subsection{Database searches (sequence homology)}
\label{sec-4-2}
\subsubsection{Theoretical}
\label{sec-4-2-1}

Blast and Blat. Something about how they work, and what they are used for. Are there
any alternatives?
Different types of Blast and how they relate to the type of analysis to be carried out.
\subsubsection{Practical}
\label{sec-4-2-2}

Run some blast analyses, and look at the results.
Analyse an unknown sequence, follow by suggestions for futher analysis.
\subsection{Multiple sequence alignment and phylogeny}
\label{sec-4-3}
\subsubsection{Theoretical considerations}
\label{sec-4-3-1}

\begin{itemize}
\item Purpose of alignment
\item Identification of important residues
\item Determination of relationships (orthology / paralogy / equivalent function)
\item Give information on protein structure
\end{itemize}
\subsection{Sequence (\& genome) assembly}
\label{sec-4-4}
\subsection{Genome annotation}
\label{sec-4-5}
\subsection{Literature (not just bioinformatics)}
\label{sec-4-6}
\subsection{Google! (everything, network structures).}
\label{sec-4-7}
\subsection{Organisation of data}
\label{sec-4-8}
\subsubsection{Structured vs non-structured}
\label{sec-4-8-1}

    The only thing I learnt from my supervisor: how to label tubes.
\subsubsection{Backends and frontends}
\label{sec-4-8-2}
\subsubsection{Relational vs object-oriented (probably not)}
\label{sec-4-8-3}
\subsubsection{Representing data structure in a relational framework}
\label{sec-4-8-4}

\begin{itemize}
\item Use Ensembl as an example of how gene structure is encoded, and how this extends
  to additional data. Eg. genes -> transcripts -> shared exons.
\item Representing laboratory data, reagents (plasmids, antibodies), experiments, procedures.\\
And why we don't actually bother doing so.
\end{itemize}
\subsubsection{Making a mess of it}
\label{sec-4-8-5}

UCSC genome browser database (maybe not worth it)
\subsection{Explaining the behaviour of systems (modelling component interactions)}
\label{sec-4-9}
\section{Genomics \& Large data sets}
\label{sec-5}

\begin{itemize}
\item Genomics: questions related to full genomes
\item Functional Genomics: high throughput technologies
\begin{itemize}
\item Fishing expeditions (common). Identify some genes likely to be involved in some process.
    Not really genomics / transcriptomics / proteomics but often called as such.
\item Genome scale characterisation (eg, how many genes, spacing, proportion functional,
    spatial organisation, genome evolutionary mechanisms, etc\ldots{})
\end{itemize}
\item History of large data sets. (Historical background)
  SAGE, Microarrays, HTS (including PacBio, Nanopore), image data \& transcript counting
\item Types of data analysed.
  Expression, DNA-protein interactions \& Histone modifications, genome assemblies (i.e.
  the structure of genomes, and their annotation).
\item How to analyse:
\begin{itemize}
\item Visualisation of data (distributions, dimensionality reduction methods, relationships),
\item derivatives
\item models (primarily linear)
\end{itemize}
visualisation -> derivatives -> visualisation of derivatives -> selection \& visualisation
  of low level data.
\end{itemize}

\end{document}
