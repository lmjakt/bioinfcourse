% Created 2015-08-12 on. 12:10
\documentclass{scrartcl}
\usepackage[utf8]{inputenc}
\usepackage[T1]{fontenc}
\usepackage{fixltx2e}
\usepackage{graphicx}
\usepackage{longtable}
\usepackage{float}
\usepackage{wrapfig}
\usepackage{soul}
\usepackage{textcomp}
\usepackage{marvosym}
\usepackage{wasysym}
\usepackage{latexsym}
\usepackage{amssymb}
\usepackage{hyperref}
\tolerance=1000
\providecommand{\alert}[1]{\textbf{#1}}

\title{Study objectives for B1229F 2015 (Genomikk og Bioinformatikk)}
\author{Lars Martin Jakt\thanks{lmj@uin.no}}
\date{\today}
\hypersetup{
  pdfkeywords={},
  pdfsubject={},
  pdfcreator={Emacs Org-mode version 7.9.3f}}

\begin{document}

\maketitle


\section{The central dogma}
\label{sec-1}

You will gain familiarity with how genetic information is encoded within
the genome and how this is interpreted by the components of the cell.
\section{Bioinformatics and -omics}
\label{sec-2}

You will gain an overview of what bioinformatics entails, learn how to
perform basic analyses and gain a basic understanding of how these analyses
work: 
\begin{itemize}
\item make use of public resources to 
  obtain DNA and RNA sequences and how to extract information
  from these.
\item a basic understanding of sequence alignment and sequence similarity.
\item the meaning and evolutionary origins of orthology and paralogy.
\item visualise large data sets, and how to
  find both overall patterns and specific features.
\end{itemize}
\section{Informatics}
\label{sec-3}

You will gain an understanding of the relationship between data and information,
and how this relates to biological understanding.
\section{Statistics}
\label{sec-4}

You will be introduced to and gain familiarity with:
\begin{itemize}
\item The use and limitations of summary values (eg. averages \& variances).
\item The relationship between distributions and probabilities
  and how this is used to infer statistical significance.
\item The limitations of statistical inference and
  its interpretation.
\item How to make and evaluate your own derivative statistics.
\end{itemize}
\section{Computational skills}
\label{sec-5}

You will gain familiarity with the relationships between:
\begin{itemize}
\item Computer operating systems
\item Operating system environments (shells)
\item Applications
\item Networks and file systems
\item Clients and servers.
\end{itemize}
\section{Systems biology}
\label{sec-6}

Gain familiarity with how biological pheonomena can be described and modelled
as systems of interacting entities.

\vspace{1.5cm}
Most importantly:
\par\bigskip {\Large\bfseries\noindent
  Become able to extend your knowledge independently
}\par\smallskip



\end{document}
