% Created 2015-07-31 fr. 13:09
\documentclass{article}
\usepackage[utf8]{inputenc}
\usepackage[T1]{fontenc}
\usepackage{fixltx2e}
\usepackage{graphicx}
\usepackage{longtable}
\usepackage{float}
\usepackage{wrapfig}
\usepackage{soul}
\usepackage{textcomp}
\usepackage{marvosym}
\usepackage{wasysym}
\usepackage{latexsym}
\usepackage{amssymb}
\usepackage{hyperref}
\tolerance=1000
\providecommand{\alert}[1]{\textbf{#1}}

\title{Timetable for B1229F 2015}
\author{Lars Martin Jakt}
\date{\today}
\hypersetup{
  pdfkeywords={},
  pdfsubject={},
  pdfcreator={Emacs Org-mode version 7.9.3f}}

\begin{document}

\maketitle

\setcounter{tocdepth}{3}
\tableofcontents
\vspace*{1cm}

The term time has 13 weeks (34-46), with 2 2 hour sessions scheduled per week.
However, due to external meetings (on weeks 38, 41 and 44 Monday), 
11 and a half weeks are available.

\begin{enumerate}
\setcounter{enumi}{33}
\item 
\begin{itemize}
\item Mon: General Introduction to the course and to what
     is expected of the students. Use second half to understand the students' background.
\item Wed: Central Dogma and the flow of information.
\end{itemize}
\item 
\begin{itemize}
\item Mon: Bioinformatics, general introduction to the problem domain. (Why we spend so much time on sequences).
\item Wed: Biological databases and sequence formats.
\end{itemize}
\item 
\begin{itemize}
\item Mon: Pairwise Sequence alignment (substitution matrices, etc).
\item Wed: Multiple sequence alignment and phylogeny (relationship to evolution).
\end{itemize}
\item 
\begin{itemize}
\item Mon: Informatics. Practical aspects (computers, operating systems, file systems, applications) 1.
\item Wed: Informatics. Practical aspects (computers, operating systems, file systems, applications) 2.
\end{itemize}
\item 
\begin{itemize}
\item Mon: Leave
\item Wed: Leave
\end{itemize}
\item 
\begin{itemize}
\item Mon: Practical sequence alignment. Web based demonstrations / practicals?
\item Wed: Database searches (Blast, Blat, etc), theory \& practical considerations.
\end{itemize}
\item 
\begin{itemize}
\item Mon: Mathematical considerations 1. Summary statistics (derivatives): means \& variances.
\item Wed: Mathematical considerations 2. Linking distributions to probabilities.
\end{itemize}
\item 
\begin{itemize}
\item Mon: Regulation of transcription (Promoters, trans factors, cis acting elements, chromatin).
\item Wed: Cells as systems 2. Integration of input and output with state.
\end{itemize}
\item 
\begin{itemize}
\item Mon: Leave.
\item Wed: Leave.
\end{itemize}
\item 
\begin{itemize}
\item Mon: Genomics / Transcriptome / Proteome.
\item Wed: Omics: purpose and methods?
\end{itemize}
\item 
\begin{itemize}
\item Mon: Leave
\item Wed: Visualising and analysing large data sets.
\end{itemize}
\item 
\begin{itemize}
\item Mon: Theoretical. Akaike's information content.
\item Wed: Data to information the biologist's perspective. (Opposite)
\end{itemize}
\item 
\begin{itemize}
\item Mon: Databases, esp. relational databases.
\item Wed: Databases, looking at Ensembl structure.
\end{itemize}
\end{enumerate}

I should also squeeze in a section on looking at large data sets This should include
things like:
\begin{enumerate}
\item Data structure
\item Overall patterns (similar to structure,  but related to biological questions)
\item Data mining (looking for specific things).
\end{enumerate}

I would also like to squeeze in a bit of R in doing that. That is likely to take
rather more time than a 4 hours in a week.

The order of the above should also be modified.

\begin{enumerate}
\item 
\begin{itemize}
\item hello
\item goodbye
\end{itemize}
\item 
\begin{itemize}
\item hello2
\item and something else
\end{itemize}
\end{enumerate}

\end{document}
