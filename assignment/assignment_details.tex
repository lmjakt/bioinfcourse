\documentclass[11pt]{article}
\usepackage[utf8]{inputenc}
\usepackage[T1]{fontenc}
\usepackage{fixltx2e}
\usepackage{graphicx}
\usepackage{longtable}
\usepackage{float}
\usepackage{wrapfig}
\usepackage{rotating}
\usepackage[normalem]{ulem}
\usepackage{amsmath}
\usepackage{textcomp}
\usepackage{marvosym}
\usepackage{wasysym}
\usepackage{amssymb}
\usepackage{capt-of}
\usepackage{hyperref}
\tolerance=1000
\usepackage{minted}
\usepackage{color}
\usepackage{listings}
\usepackage{grffile}
\usepackage[inline]{enumitem}
\usepackage{xcolor}
\hypersetup{
colorlinks,
linkcolor={red!50!black},
citecolor={blue!50!black},
urlcolor={blue!80!black}
}
\usepackage{setspace}%% The linestretch
\singlespacing
\usepackage[format=hang,indention=0cm,singlelinecheck=true,justification=raggedright,labelfont={normalsize,bf},textfont={normalsize}]{caption} %
\usepackage{vmargin}
\setpapersize{A4}
\setmarginsrb{2.5cm}{1cm}% links, oben
{2.5cm}{2cm}% rechts, unten
{12pt}{30pt}% Kopf: Höhe, Abstand
{12pt}{30pt}% Fuß: Höhe, AB
\usepackage{upquote}
%  use straight quotes when printing a command in minted
\AtBeginDocument{%
\def\PYZsq{\textquotesingle}%
}
\definecolor{mintedbackground}{rgb}{0.95,0.95,0.95}
\setlength{\parindent}{0pt}
\setlength{\parskip}{\baselineskip}
\definecolor{mintedbackground}{rgb}{0.95,0.95,0.95}
\author{Martin Jakt\thanks{University of Nordland, Norway}}
\date{\textbf{Bioinformatics \& Genomics}}
\title{\textbf{BI229F Assignment} (2016)}
\hypersetup{
 pdfkeywords={},
  pdfsubject={},
  pdfcreator={}}
\begin{document}

\maketitle
%\tableofcontents

\section{The assignment}
\label{sec-1}

Your assignment is to write a somewhat useful program or script in Perl. Below
you can find a list of tasks that I consider suitable. These tasks are divided
by the difficulty of the implementation; the simpler the task the lower the
grade that will be awarded. If you can think of a different task that you
would like to solve, then please discuss this with me and I'll let you know if
I consider it to be suitable.

Plesse make sure to read the two last sections on what you need to hand in,
and how you will be marked.

\emph{You are free to ask me for advice while you are writing the code and the
  report; and I would recommend you to make use of this well before the
  deadline for the report.}

\subsection{Suitable tasks}
\subsubsection{Difficult tasks}
A more or less working solution to one of these tasks will be rewarded with an
A, even if it is not so beautifully implemented.
\begin{itemize}
\item Implement the core of the BLAST algorithm producing simply ungapped
  alignments. This is quite difficult, but you can make it simpler, by writing
  code that is specific for a given word length. (Note that a hash in perl is
  a lookup table.)
\item Given a SAM file and the target genome sequence, identify and enumerate
  sequence variants.
\item Any sequence alignment algorithm where the output is given in a
  graphical way (either for the alignment, or to visualise the path through
  the scoring matrix). This gets points for working out how to create graphics,
  which we did not cover in much detail.
\end{itemize}

\subsubsection{Intermediate tasks}
A working solution will be given a B. An A may be given if the script shows
innovation in some way; eg. in how it outputs the results and or interacts
with the user.
\begin{itemize}
\item Sequence alignment by a dynamic programming algorithm. This should not
  be a straight Needleman-Wunsch since we have done this in class. However, if
  you modify it some way (eg. so that it treats terminal gaps differently)
  then this is acceptable. Note that whatever the algorithm is, you need to
  provide a reasonable formatted output.
\item Extract sequencs from a SAM file and convert to a different format. Note
  that sequences in a SAM file can be reverse-complemented, so for this you
  need to be able to interrogate the alignment flag correctly and to reverse
  complement sequences.
\item Count the occurences of words (sub-sequences of a specific length) in
  a set of sequences and report the distribution.
\item A dotplot with variable word sizing and graphical output using something
  like SVG.
\end{itemize}

\subsubsection{Easy tasks}
Working solutions to these will be given a C. Again the score can be boosted
by doing something interesting in the script.
\begin{itemize}
  \item Convert fasta sequence data to some other format (eg. tab-delimited),
    possibly splitting into individual files.
  \item Convert fastq data to fasta.
  \item Reverse complement a sequence.
\end{itemize}


\subsection{What to do when you're stuck}
You will probably find yourself stuck with errors that you don't understand,
or can't find the cause of at some point; or perhaps you simply can't work out
what to do next. When this happens, I suggest the following:
\begin{itemize}
\item Read the error and warning messages; they usually tell you where the
  problem might be in the code. Occasionally the error message will point you
  in the wrong direction, so don't spend too much time staring at a single
  line of code. If you don't understand the message, you can always try
  googling it.
\item Experiment with the code. When writing the code always run it after you add
  some new component or function. When doing so, print out the contents of variables so
  that you can determine what the script is doing. Change stuff in your script
  and then see what happens.
\item Talk to the other students; often you may have a missing \verb|$| or a
  \verb|$| instead of a \verb|@| in your code that you just cannot see
  yourself. A second pair of eyes will often pick this up much faster than
  you. Talking about things also tends to make you think of things in a
  different way.
\item Talk to me.
\end{itemize}

Don't spend too much time stuck in a line of code before you do some of the above.

\section{What you need to hand in}
You will need to hand in a report containing two or three parts:
\begin{itemize}
\item A text file containing the source code to your script. This should be a
  working script that I can  run on my computer.
\item Optionally; a set of input files that work with the script. If, for
  example, your script aligns two sequences you may wish to provide a file or
  files that the script can read as input.
\item A report describing what your script does. This should include:
  \begin{itemize}
  \item A section outlining what your script does and why it is useful.
  \item A more detailed description of the algorithm, or how your script
    solves its task.
  \item A description of how to use your script; i.e. what input does it
    require (including the formats of input data) and how it outputs the results. You
    can think of this as writing the manual page for the program.
  \item Description of an example run, and an appraisal as to how well your
    script works.
  \end{itemize}
\end{itemize}

The work will need to be submitted to fronter. The instructions on fronter
state that you must submit a single pdf file. Please ignore this and instead
submit a compressed archive file (eg. a zip or tar.gz file). Give the file a
name that includes your student number. Note that you may have different
student numbers for the assignment and the exam.

\section{How your assignments will be graded}
You will be marked on both the quality of your script and the accompanying
report.
\subsection{The script}
Your script should work. There are several parts to a script working:
\begin{itemize}
\item The script runs without errors or warnings when given appropriate input
  data.
\item The script produces the intended result, eg. for a dynamic programming
  algorithm it should produce one or more optimal alignments.
\item How well the script works with badly formatted input. What a program
  should do when confronted with data that is incorrectly formated is not
  always clear; however, the worst thing it can do is to continue as normal
  and then output incorrect output. Regardless of what the script does it
  should be described accurately in the report, eg.:
  \begin{itemize}
  \item This script does not check for incorrectly formatted input; the
    results of using bad input are undefined.
  \item This script will die if it believes the input is not a correctly
    formated fasta file.
  \end{itemize}
  are both acceptable descriptions. The second option describes a better
  program then the first; but both can be correct.
\end{itemize}

If you have problems with your script doing what it should do, then your best
option is to make it work, but if at the deadline it is not working as you
want it to, then you need to explain what parts of your script are working, 
where it fails, and what you think the problem is. Note that you
need to allocate enough time to write the report.

Sometimes a script will work with one set of input, but not another. This
usually indicates a bug in the logic. There are often such bugs in code; for
this reason it can be a good idea to include in your report input data which
you know will work with the script. Otherwise I may end up trying to run your
script and conclude that it doesn't work. If at the same time you claim that
it works, I may end up suspecting a certain dishonesty on your part. That
would be bad.

Points will be awarded and deducted by the following criteria:
\begin{itemize}
\item Plus points:
  \begin{itemize}
  \item Correctness: i.e. doing what it says on the tin.
  \item Novelty: doing things not described by me in the course.
  \item Style: the script should be easy to read and understand.
  \item Flexibility: how does the script deal with bad data. Will it silently do
    the wrong thing (bad) or complain about the data (good).
  \item Efficiency: the script should do as little work as possible to achieve
    its task. However, efficiency here is much less important than readability
    and clarity.
  \end{itemize}
\item Minus points:
  \begin{itemize} 
    \item Not working (this is a big minus).
    \item Doing exactly the same things as the scripts introduced in the
      lectures and course notes.
    \item Looking the same; i.e. having the same structures and variable
      names as the scripts I've shown you. You \emph{will} be penalised for
      having all the same variable names as I have used in my lectures.
  \end{itemize}
\end{itemize}

In essence you will be awarded if I think you worked out the logic of the
script yourself and penalised if I think that you simply copied a load of
stuff and made it work without understanding it.

\subsection{The report}

The purpose of the report is partly to give you a chance to convince me that
you understand what your script does; and also that you wrote it. As usual
this is best achieved by using your own words, rather than borrowing phrases
from presentations or other documents. I do not wish to see a very long
report, but the length will clearly depend on the complexity of your script.

The most important aspect of your report is clarity and accuracy. Do not
include more words than necessary. Absolutely do not state things which are
not completely true; do not overstate what you have done. If your script
mostly works, but there is a problem you are aware of; then describe
this. It is unlikely that there is no such problem, so you will be given
points for being aware of when things may not work. If you write, 'my script
works beautifully', when quite clearly it does not, you will be \emph{heavily}
penalised. 

I expect that a suitable report will be of a similar length to this document,
but you should not aim for a specific length.

\end{document}
