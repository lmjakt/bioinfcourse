\documentclass[11pt]{article}
\usepackage[utf8]{inputenc}
\usepackage[T1]{fontenc}
\usepackage{fixltx2e}
\usepackage{graphicx}
\usepackage{longtable}
\usepackage{float}
\usepackage{wrapfig}
\usepackage{rotating}
\usepackage[normalem]{ulem}
\usepackage{amsmath}
\usepackage{textcomp}
\usepackage{marvosym}
\usepackage{wasysym}
\usepackage{amssymb}
\usepackage{capt-of}
\usepackage{hyperref}
\tolerance=1000
\usepackage{minted}
\usepackage{color}
\usepackage{listings}
\usepackage{grffile}
\usepackage[inline]{enumitem}
\usepackage{setspace}
\usepackage{tikz}
\usepackage{subcaption}
\usepackage{xcolor}
\usepackage{fancyvrb}
\hypersetup{
colorlinks,
linkcolor={red!50!black},
citecolor={blue!50!black},
urlcolor={blue!80!black}
}
\usepackage{setspace}%% The linestretch
\singlespacing
\usepackage[format=hang,indention=0cm,singlelinecheck=true,justification=raggedright,labelfont={normalsize,bf},textfont={normalsize}]{caption} %
\usepackage{vmargin}
\setpapersize{A4}
\setmarginsrb{2.5cm}{1cm}% links, oben
{2.5cm}{2cm}% rechts, unten
{12pt}{30pt}% Kopf: Höhe, Abstand
{12pt}{30pt}% Fuß: Höhe, AB
\usepackage{upquote}
%  use straight quotes when printing a command in minted
\AtBeginDocument{%
\def\PYZsq{\textquotesingle}%
}
\definecolor{mintedbackground}{rgb}{0.95,0.95,0.95}
\setlength{\parindent}{0pt}
\setlength{\parskip}{\baselineskip}
\definecolor{mintedbackground}{rgb}{0.95,0.95,0.95}
\definecolor{mintedBg}{rgb}{0.95, 0.95, 0.95}
\definecolor{blockBg}{rgb}{0.6, 0.6, 0.95}

%% define styles for different codes
\newminted{cpp}{linenos, bgcolor=blockBg, fontsize=\footnotesize}
%% then use \begin{cppcode}
\newminted{c}{linenos, bgcolor=mintedBg, fontsize=\footnotesize}
\newminted{perl}{linenos, bgcolor=mintedBg, fontsize=\footnotesize}
\newminted{r}{linenos, bgcolor=mintedBg, fontsize=\footnotesize}


%% I detest indentation in footnotes etc, so try this:
\makeatletter
\renewcommand\@makefntext[1]{\noindent\makebox[0em][r]{\@makefnmark}\footnotesize#1}
\makeatother
%% the makeatletter and makeatother are required to allow me to
%% to change the macro beginning with an @. (though when I call it
%% I don't use the @ ... 

%% this should handle non-breaking space
%% characters.
\DeclareUnicodeCharacter{00A0}{~}

\renewcommand\scriptsize\normalsize

\author{Martin Jakt\thanks{University of Nordland, Norway}}
\date{\textbf{Bioinformatics \& Genomics}}
\title{\textbf{Course assignment} (2015-11-06)}
\hypersetup{
 pdfkeywords={},
  pdfsubject={},
  pdfcreator={}}
\begin{document}

\maketitle
%\tableofcontents

\section{pre-amble}
\label{preamble}
This is an exercise in looking at large scale expression data. It is very
similar to what we did in the class previously, but with a simpler data set
which should not cause you any trouble.

To perform the analysis you will need a computer with R and a network connection.
Because not all of you may have access to your own computer you can perform the
analysis as pairs or threesomes if you so wish. You should however hand in a report
individually, though in this case the source code as well as the figures will
obviously be the same.

The report should be a minimal document whose purpose is to demonstrate that you
have performed the analysis and understand (mostly at least) what it is that you
have done. The deadline for the report has been set for the 18th of December. For 
someone familiar with R the tasks outlined here should take a very short time, but
for those less familiar the exercise may take a little longer.

Read this document in its entirety before you start doing anything. You will find
more than you need in the previous document \texttt{bigDataSession.pdf}, and in
the slides for lecture 10.

\section{the GDS4805 data}
\label{datasec}
The GDS4805 data set contains expression values taken from
a Mouse Embryonic Fibroblast (MEF) cell line transfected with
4 different combinations of transcription factors including
one control sample transfected with only the \emph{M2rtTA} transcription
factor as a negative control\footnote{M2rtTA is a synthetic transcription
factor created by combining parts from several different proteins. It is used
as part of system that provides the ability to induce expression of transgenes by the addition
of tetracycline or doxycycline to cells. It should have no effect on the cells on
its own.}.

\begin{tabular}{l|l}
Sample group & Transgenes \\
\hline
1 & MrtTA \\
2 & M2rtTA, GATA4, TBX5, MEF2C \\
3 & M2rtTA, GATA4, TBX5, MEF2C, Mesp1, SMARCD3 \\
4 & M2rtTA, GATA4, TBX5, MEF2C, Mesp1, SMARCD3, MYOCD, SRF \\
\end{tabular}

There are three replicates for each sample group.

You can get all of the data for this at :\\
\url{http://www.ncbi.nlm.nih.gov/geo/}\\

by searching for GDS4805 and downloading the GDS4805\_full.soft file that
contains information about the samples, the probes of the array and
the expression data itself.

Since this gives you the data in a \texttt{.gz} format which some of you may have
trouble decompressing I have also made it available at:\\
\url{http://martin.jakt.org.uk/bioinfgenomics/}\\
{\bfseries \emph{Don't}} left click the link to the file, but right-click and select
'download as' and save the file (Internet Explorer can't handle the file, and will
probably die if you try to load the file directly).

\section{The tasks}
Given this data I want you to:

\begin{enumerate}
\item Load the data into an R-session.
\item Work out which samples represent which
  treatment group and make a data structure which summarises this.
  In the previous time we used a matrix, but since there is only
  one variable type (the set of transcription factors) you can use
  a vector this time.
\item Look at the distributions of the expression values using either
  a heatmap, or simply a boxplot. See if there are any outliers.
\item Perform a Principal Components Analysis (PCA) and plot the variances
  as well as the positions of the samples in the two first dimensions.
\item Identify genes which are expressed more highly in one of the treatment
  types by calculating a suitable statistic.
\item Use the information provided in the soft file to identify the the top 
  10 over-expressed genes based on the statistic
  you prepared in the previous section. Plot the expression of these
  genes.
\end{enumerate}

\section{Hints}
\subsubsection{General}
I will require you to give me the R-code that you used to perform
your analysis. Hence you will need to use a text-editor of some
sort to write the code in. The easiest way for you to do this is
to use RStudio. If you use RStudio remember to start with creating
a new Project (preferably in a new directory) before you do anything else.
You can then type the code into the source window and use Ctl. enter
to run it. If you don't wish to use RStudio, then you can use a
text editor of your choice; but do use one that gives you syntax
highlighting and helps you to balance your brackets. In any case make
sure to start the analysis in a new directory so that you can find the
input and output files.

\subsubsection{Loading data}
You can use the \texttt{GEOquery} package we installed previously. See
\texttt{bigDataSesssion.pdf} on fronter for details of how to install
and use. 

The \texttt{getGEO} function allows you to load the data. This is most
easily done if you place the soft file in your current working directory;
if not you will need to provide the path to your copy of the file. I recommend
using a downloaded file rather than loading it directly as this allows you
to get full set of probe information and avoids any trouble with firewalls
that you may have.

The function \texttt{Meta} returns a named list
that contains the various details about the experiment. The entries
named \texttt{description} and \texttt{sample\_id} are the ones you
are most interested in.

The \texttt{Table} function will give you the actual expression data. A number
of the columns contain probe descriptions and columns whose
column names begin with \texttt{GSM} contain the expression values. I recommend
extracting the expression values into a separate matrix and setting the
rownames to the \texttt{ID\_REF} column of the dataframe returned by
the \texttt{Table} function. (We used the \texttt{grep} function last time
to extract the columns with 'GSM' in their names).

Unfortunately the expression values in the dataframe returned by \texttt{Table}
are encoded as text ('character') so you will need to convert them
to a numeric type (\texttt{as.numeric}).

\subsubsection{the sample data}
You can find the sample identifiers and descriptions in \verb|$sample_id| and
\verb|$description| entries of the description data structure. Refer to our
previous session as to how these are arranged. This time it's simpler than last
time so you do not need such a complicated strategy to work out what's going on.
(remember the use of \verb|%in%|). You can arrange the data as a simple vector;
just make sure that the order of it correlates to the column order of
the expression data.

\subsubsection{distributions}
Once you have extracted the expression values to a matrix (\texttt{as.matrix}) 
and converted to
a numeric format (\texttt{as.numeric}) remember to set the 
\texttt{colnames} and \texttt{rownames} of this matrix. This will allow
you to very easily run the \texttt{boxplot} or \texttt{heatmap} functions.

If you wish to create a heatmap, remember that you will first need to use
\texttt{hist} on all of the columns using the same set of \texttt{breaks}.
Refer back to bigDataSesssion.pdf as to how you can do that (it's a little
more complex than just using a boxplot).

Secondly remember that looking directly at the values is probably not going
to tell you very much; you will probably need to log-transform the values
(\texttt{log}).

\subsubsection{PCA}
You can do the PCA using the \texttt{prcomp} function. Remember that you will
need to transpose the table (using \texttt{t}). You can try with or without
log transformation of the data, but remember to set the \texttt{scale} option
to \texttt{TRUE} if you do not log transform (it's good in both cases, but
more so if you don't log transform).

\texttt{prcomp} returns a named list. Calling plot on this will plot the variances
associated with the dimensions. To plot the points look in the \verb|$x| component.

\subsubsection{Identifying differentially expressed genes}
There are lots of ways to identify differentially expressed genes. The simplest
scenario is to simply take the difference or ratio of the mean of two
sample classes. However, this doesn't take into account the variance within
the classes. The simplest statistic that does this is the t-statistic (difference
in means divided by the variance). R has a built-in t-test function, but to
use it on the rows of your data is a bit of a pain. You can do something like:

\begin{rcode}
## you have extracted the expression data into a matrix
## called dat, whose column names are set to the sample identifiers
## (GSMXXX, etc.). You also have the sample treatments in a vector
## with one entry for each sample, called sample.treat. You can then
## do something like

## get a logical (TRUE / FALSE) vector giving the samples which are of the
## first treatment type...
s1 <- sample.treat == sample.treat[1]

tstats <- apply(dat, 1, function(x){ t.test(x[s1], x[!s1])$statistic })
## see ?t.test to see why we add the $statistic after the call.

## to get an order that gives the biggest tstats values first we 
## use the order function:

tstats.o <- order(tstats, decreasing=TRUE)

## you can now look at the what the expression looks like using
## this index..
plot(dat[tstats.o[1],])
## though that is not a very pretty plot. You should add some colours
## to indicate the sample types.
\end{rcode}

\subsubsection{Exporting plots}
To export plots as files you simple change the display driver to
something that gives you a file. I usually use \texttt(pdf), but it
may be easier for you to use the \texttt{png}, \texttt{jpeg} or \texttt{tiff}
drivers, since I wish you to put the plots you get into a report.
The usual word processors (i.e. Word\footnote{Maybe it does by now, but I've never
managed to get Word to do the right thing with pdf files}) don't handle pdf imports very
well and there is no point to spend a load of time just to get better pictures.

\begin{rcode}
## to plot stuff to a file, simply call
## the device driver you want to use..
## eg. to make a 1024x1024 tiff do:
tiff("plotname.tif", width=1024, height=1024)
## do your plotting between the call to tiff and the
## final dev.off() call for example:
plot(1:10, (1:10)^2, pch=19)
points(c(3:5), c(3:5)^2, pch=1, cex=1.5, col='red', lwd=2)

## and then call dev.off() to finish plotting to the file.
dev.off()
\end{rcode}

If you wish to plot more than one chart on the same figure
you can use \texttt{par(mfrow=c(r, c))} where r and c represent
the row numbers. The call to \texttt{par} should be made immediately after
you open the plot device. For example...

\begin{rcode}
## here I set the width to twice the height
## since I want to put two plots on one row
pdf("ugly_plot.pdf", width=14, height=7)
par(mfrow=c(1,2))
plot(1,1, pch=1, cex=20)
plot(1,1, pch=19, cex=20)
dev.off()
\end{rcode}

\section{What you should hand in}
A file (text format, or possibly as a pdf, but raw text
is the best) containing the code that you used to perform
the analysis.

A document (preferably as a pdf) containing a description of what
you did; the plots you produced, and any conclusions you can make about the
data (eg. whether the replicates are good, whether there are clear differences
in gene expression between the different treatment groups).

\emph{This document should include the names of the group or individual who performed
the analysis.}

This should not be long; aim to be concise and clear; use an active voice,
'I/we looked at the distribution of the log expression values using the hist() function.
All distributions appeared similar with no obvious outliers.'

Don't write, 'an analysis was performed, ...', and try to minimise the use
of technical words when not needed.

\end{document}