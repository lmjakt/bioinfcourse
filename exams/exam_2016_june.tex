\documentclass[11pt]{article}
\usepackage[utf8]{inputenc}
\usepackage[T1]{fontenc}
\usepackage[a4paper]{geometry}
\usepackage{comment}
\usepackage{minted}
\usepackage{multirow}
\usepackage{enumerate}
\usepackage{tikz}
\usepackage{enumitem}

%% for listings..
\definecolor{mintedBg}{rgb}{0.95, 0.95, 0.95}
\definecolor{blockBg}{rgb}{0.6, 0.6, 0.95}
\newminted{perl}{linenos, bgcolor=mintedBg, fontsize=\footnotesize}
\newminted{r}{linenos, bgcolor=mintedBg, fontsize=\footnotesize}
\newminted{console}{linenos, bgcolor=mintedBg, fontsize=\footnotesize}

\specialcomment{Notes}{%
  \begingroup\small\color{red}}{%
  \endgroup}
%\excludecomment{Notes}

\setcounter{page}{2}

\begin{document}

This exam contains a total of 8 sections. The majority of these sections have
some choice in the questions that you answer. Please read the instructions
below each section head. A total of 100 marks will be considered as full marks.

The number of marks available from each section is indicated in the section
headings. Note that it is possible to select questions that total more than
that mark; however, you will not be able to get more marks from a given
section by answering more questions. If you find that your answer to a given
question is no good and you wish to change your selection, then cross out the
old one to indicate which one you do not wish to be marked for.

Feel free to illustrate your answers wherever you feel appropriate; but do
remember to include labels so that I can decipher your figures. Do not include
a picture without some descriptive text. Two of the questions will need to be
filled in on a supplementary sheet. Do not forget to hand this in with your
other answers.

Marks have been assigned on the assumption of a four hour exam; that is one mark
corresponds to 2.4 minutes. For some of the questions that may mean mostly
writing, but for others (esp. code and manual snippets) that time may be taken up by
reading the details of the question, and the answer itself may be very
short.


\section{Bioinformatics in general (12 points)}
\begin{enumerate}
\item Describe an example where you would need to use a bioinformatics
  analysis. Include both a description of the problem you are concerned with
  and an outline of the analysis you would perform. About half of your answer
  should describe the problem your example is concerned with; the other half
  should describe the approach to solving the problem. Do not go into too much
  detail!\\
  (6 points)\\
\begin{Notes}
  This is an open ended question with no specific answers. I do not expect a
  great deal of detail (eg. specific versions of programs used), but the
  student can receive marks for reasonable answers even if these are not
  technically correct.
\end{Notes}

\item Sequencing:
  \begin{itemize}
    \item How does sequence data differ from (most) other types of data obtained
      in biological research?\\
      (2 points)\\
\begin{Notes}
  Sequence data is qualitative; most other data types are quantitative. This
  makes it easier to define identities. \\
  Students can also get points for other
  reasonable answers, such as the relative cost of obtaining sequence data,
  information density and other such esoteric issues.
\end{Notes}
    \item Describe how sequencing technology has progressed from standard
      Sanger sequencing to the latest next-generation sequencing.\\
      (4 points)\\
\begin{Notes}
  From few long sequences to large numbers of very short (30-50) sequences, to
  very large numbers of intermediate length sequencing (300-600) to the
  latest single molecule based methods giving extremely long (50,000)
  reads. Points for giving names of technologies, their capacities and
  lengths, but I do not expect all technologies or exact numbers to be
  recalled; the most important point is to give the relative trend in
  terms of capacity and read length.
\end{Notes}
(total 6 points)\\
  \end{itemize}
\item In the literature you will frequently see the terms 'genomics' and
  'transcriptomics'. What is meant by these terms? Give two examples of the type
  of questions addressed by both.
  (6 points)\\
\begin{Notes}
  Genomics and transcriptomics: the study of complete genomes and or
  transcriptomes; i.e. the study of the full set of DNA sequences (genomes) or
  RNA sequences (transcriptome) present in an organism (genome) or cell type
  (transcriptome). Questions may address the properties of the full set of
  data, eg. distribution of genes and repetitive sequences across the genome,
  and or the distribution of gene expression levels. However, the terms are
  most frequently used to describe data mining operations where the activity
  or influence of small number of individual gene or gene features are
  identified from the full 'ome. The student may give any reasonable example
  of question addressed (eg. changes in gene expression induced by external
  factors, identification of mutations associated with a phenotype, comparison
  of whole genomes to infer evolutionary history, etc.).
\end{Notes}
\end{enumerate}

\section{Molecular Biology \& the Central Dogma (20 points)}
\begin{enumerate}
\item Describe how proteins are encoded in genomes and how this code is read
  to synthesize specific proteins. Include a description of
  how this differs in eukaryotes and prokaryotes as well as details about the
  encoding of amino acids.\\
  (10 points)
\begin{Notes}
  Encoding of amino acid in triplet code; allows encoding of 64 identities,
  but only 20 amino acids; hence redundant. Doublet code too short for 20
  amino acids. Eukaryotes, code for single protein split across several exons
  seperated by introns in a single transcription unit. Prokaryotes, code for
  several proteins contained in a single multicistronic message.
  Transcription to RNA, export (in eukaryotes), translation at ribosomes
  through the assembly of tRNA.
\end{Notes}

\item Given a genetic code:\\
  \begin{minipage}{0.6\textwidth}
  {\tiny
    %% this requires 
%% usepackage{multirow}
%% usepackage{tabularx}

\renewcommand{\arraystretch}{1.25}
\begin{tabular}{ |l| l l|l l| l l|l l|l| }
  \hline
  \multirow{2}{2em}{1st base} &
  \multicolumn{8}{|c|}{2nd base} &
  \multirow{2}{2em}{3rd base} \\
  \cline{2-9}
  &
  \multicolumn{2}{|c|}{U} &
  \multicolumn{2}{|c|}{C} &
  \multicolumn{2}{|c|}{A} &
  \multicolumn{2}{|c|}{G} & \\
  \hline
  \multirow{4}{2em}{U} & 
  UUU & \multirow{2}{4em}{\tiny (Phe/F)} &
  UCU & \multirow{4}{4em}{\tiny (Ser/S)} &
  UAU & \multirow{2}{4em}{\tiny (Tyr/Y)} &
  UGU & \multirow{2}{4em}{\tiny (Cys/C)} & U \\
  & UUC & & UCC & & UAC & & UGC & & C \\ \cline{2-3} \cline{6-9}
  & UUA & \multirow{6}{4em}{\tiny (Leu/L)} 
  & UCA & & UUA & Stop & UGA & (Stop) & A \\ \cline{8-9}
  & UUG & & UCG & & UAG & Stop & UGG & \tiny (Trp/W) & G \\ \cline{4-9} 
  \cline{1-1}
  \multirow{4}{2em}{C}
  & CUU & & CCU & \multirow{4}{4em}{(Pro/P)} & CAU & \multirow{2}{4em}{(His/H)} & CGU & \multirow{4}{4em}{(Arg/R)} & U \\
  & CUC & & CCC & & CAC & & CGC & & C \\ \cline{6-7}
  & CUA & & CCA & & CAA & \multirow{2}{4em}{Gln/Q} & CGA & & A \\
  & CUG & & CCG & & CAG & & CGG & & G \\ \cline{1-9}
  \multirow{4}{2em}{A}
  & AUU & \multirow{3}{4em}{(Ile/I)} & ACU & \multirow{4}{4em}{(Thr/T)} & AAU 
  & \multirow{2}{4em}{(Asn/N)} & AGU & \multirow{2}{4em}{(Ser/S)} & U \\
  & AUC & & ACC & & AAC & & AGC & & C \\ \cline{6-9}
  & AUA & & ACA & & AAA & \multirow{2}{4em}{(Lys/K)} & AGA & \multirow{2}{4em}{(Arg/R)} & A \\
  \cline{2-3}
  & AUG & (Met/M) & ACG & & AAG & & AGG & & G \\ \cline{1-9}
  \multirow{4}{4em}{G}
  & GUU & \multirow{4}{4em}{(Val/V)} & GCU & \multirow{4}{4em}{(Ala/A)} 
  & GAU & \multirow{2}{4em}{(Asp/D)} & GGU & \multirow{4}{4em}{(Gly/G)} & U \\
  & GUC & & GCC & & GAC & & GGC & & C \\ \cline{6-7}
  & GUA & & GCA & & GAA & \multirow{2}{4em}{(Glu/E)} & GGA & & A \\
  & GUG & & GCG & & GAG & & GGG & & G \\ \hline
\end{tabular}

  }
  \end{minipage}

  Given the following (double stranded!) genomic sequence:\\
  \verb|5' GCGATGGGCGGTGGGGTG 3'|\\
  determine all possible translations.\\
  (4 points)

\begin{Notes}
  6 translations, 3 forward, 3 reverse, beginning with 'Ala,Met,...'
\end{Notes}

\item What kind of mutations can arise in DNA sequences? Explain how the
  consequences of these mutations differ depending on the type of mutation and
  where in the genome the mutation happens.\\
  (6 points)

\begin{Notes}
  substitutions, insertions / deletions, translocations. Substitutions inside ORFs
  can lead to changes in protein sequence and premature stops, inserts /
  deletions lead to frame shift. Translocations can lead to gene fusions
  and changes in gene expression. Point mutations outside of genes generally
  have no effect.
\end{Notes}

\item How do the monomers of nucleic acids differ from those of proteins? How
  does this relate to their function and to the fact that DNA is usually double-stranded?\\
(6 points)

\begin{Notes}
  The monomers of nucleic acids are structurally similar to each other; this
  means that the sequence of a nucleic acid molecule has little effect on its
  structure. The double-stranded nature of DNA decreases the structural
  influence of the primary sequence. The primary function of nucleotide
  polymers is to encode information; this function would be compromised if the
  sequence also had a strong effect on physical structure. In contrast the
  function of proteins arises from their structure and this is facilitated by
  the variety of chemical structures present in different amino acids.
\end{Notes}
\end{enumerate}

\section{ Biological databases \& formats (8 points) }
\begin{enumerate}
\item Given an approximatly 1000 base-pair long nucleotide sequence what would
  you do to identify the source and function of the sequence? Give some
  details as to what types of databases you would use.\\
  (8 points)\\

\begin{Notes}
  Search nucleotide databases for similar or identical sequences; determine if
  sequence is likely to be transcribed and or translated by looking at the the
  sequences it matches. If transcribed, then find function from genome
  databases like ensembl and species functional databases (eg. mouse genome
  informatics / zfin). If no identical sequences found use closest homologue,
  and use orthology to extend functional annotation. If sequence is not
  transcribed, but genomic match is found then find adjacent genes and get
  function of those, to get an idea; also consider looking at extent of
  conservation across species (available from genome databases like
  ensembl).\\
  The above is an example of what can be done. But there are many more
  options and different ways to answer this question.
\end{Notes}

\item Describe the fastq and fasta sequence formats.\\
(4 points)

\begin{Notes}
  Indicate how headers, sequence and quality data are distinguished in these files.
\end{Notes}

\item The SAM format is used to describe the alignments of large numbers of
  sequences to a common reference sequence (typically a genome sequence). Each
  alignment is described by a single line of tab delimited text. The second
  field of this description is a FLAG that uses a single number to encode
  speficic combinations of 12 flags that indicate individual
  properties of problems of the sequences or alignments as indicated by the
  following table:

  {\small
  \begin{tabular}{rrl}
    Bit & Value & Description\\
    \hline
    1 & 1 & template having multiple segments in sequence \\
    2 & 2 & each segment properly aligned according to the aligner\\
    3 & 4 & segment unmapped\\
    4 & 8 & next segment in the template unmapped\\
    5 & 16 & SEQ being reverse complemented\\
    6 & 32 & SEQ of the next segment in the template being reverse complemented\\
    7 & 64 & the first segment in the template\\
    8 & 128 & the last segment in the template\\
    9 & 256 & secondary alignment\\
    10 & 512 & not passing filters, such as platform quality controls\\
    11 & 1024 & PCR or optical duplicate\\
    12 & 2048 & supplementary alignment\\
  \end{tabular}
  }

  Each 'Value' in the above table is an exact power of 2 which means that it
  is encoded in binary by a single 1 in the indicated position. A single
  number can thus hold any combination of the given flags. For example, if a
  given sequence failed to pass filters, and has been reverse complemented we
  set the 10th (512) and 5th (16) bits to 1 giving us 528 (512 +
  16) which is represented in binary as: \verb|001000010000|.\\ 
  Give the binary encodings and describe the meanings of the following 
  FLAG values: 147, 99, 98. Which one of these does not make sense and why?\\ 
  (4 points)

\begin{Notes}
  147 = 128 + 16 + 2 + 1 = 000010010011 $\Rightarrow$ multiple segments,
  each segment properly aligned, sequence reverse complemented, last segment
  in template\\
  99 = 64 + 32 + 2 + 1 = 000001100011 $\Rightarrow$ multiple segments,
  each segment properly aligned, sequence of next segment reverse
  complemented, first segment in the template\\
  98 as 99, but flag for multiple segments not set; hence does not make
  sense as the flags refer to other segments in the template.
\end{Notes}

\end{enumerate}

\section{Pairwise alignment (20 points)}
\begin{enumerate}
\item Given the two sequences \verb|5' GCTATGA 3'| and \verb|5' GTTTA 3'|\\
  \begin{itemize}
    \item Provide at least two reasonable alignments
    \item Describe how you can score the alignments using a set of penalties
      or scores.
    \item Score the alignments using your own scoring system. Which of your
      alignments is the best and how does this relate to your scoring system?
    \item Are your alignments global or local? Describe how this affects the
      score.
    \item Describe what is meant by an 'affine gap penalty'. Does it have any
      effect on your alignments?
  \end{itemize}
  (10 points)

\begin{Notes}
  Any reasonable alignment is OK. These should be scored by match scores, 
  and penalties for mismatches and gap insertions. Generally the hightest
  scoring alignment is the best; but changing the scoring system can change
  which has the highest score. The alignments are global if gap penalties
  have been included for terminal gaps; otherwise local. Global gives lower
  scores. Affine gap penalty: different penalties for gap insertion and gap
  extension. In this case unlikely to have any effect as no reasonable way
  to insert a double gap (as far as I can see anyway).
\end{Notes}



\end{enumerate}

\end{document}
