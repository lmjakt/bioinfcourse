%% the following is necessary for using colours in tabular
\PassOptionsToPackage{table}{xcolor}
\documentclass[pdf]{beamer}
\mode<presentation>{}
\usepackage{minted}
\usepackage{tikz}
\usepackage{pgffor} %% gives looping with \foreach
\usepackage[absolute,overlay]{textpos}
\usepackage{lmodern} %% scalable latin characters
\usetikzlibrary{arrows,shapes,backgrounds}
\usepackage{multirow}
\usepackage{tabu}
\usepackage[utf8]{inputenc}

%% it seems that beamer and descriptions can be a bit
%% tricky, so we need to define some things ourselves
\defbeamertemplate{description item}{align left}{\insertdescriptionitem\hfill}

%% I detest indentation in footnotes etc, so try this:
\makeatletter
\renewcommand\@makefntext[1]{\noindent\makebox[0em][r]{\@makefnmark}\tiny#1}
\makeatother
%% the makeatletter and makeatother are required to allow me to
%% to change the macro beginning with an @. (though when I call it
%% I don't use the @ ... 


\title{Course content and timetable}
\subtitle{B1229F Autumn 2016}
\author{Martin Jakt}

%% \setlength{\parskip}{0.5em}

\begin{document}

\begin{frame}{Lectures / classes}
  \tiny
  \begin{enumerate}
  \item General introduction
  \item Central Dogma
  \item Bioinformatics / Genomics introduction
  \item Biological databases / resources
  \item Pairwise sequence alignment
  \item Multiple alignment
  \item Database search algorithms
  \item Computers, OSes, filesystems
  \item General introduction to writing programs
  \item Introduction to Perl
  \item Practical: Sequence alignment in Perl (1)
  \item Practical: Sequence alignment in Perl (2)
  \item Using tools: online / offline sequence analysis (1) \\
    sequence identification, alignment, phylogeny
  \item Using tools: online / offline sequence analysis (2)
  \item Introduction to R
  \item Summarising numbers and deriving statistics
  \item Looking at big data ($\thicksim$ omics)
  \item Statistical considerations for big data
  \item Experimental data
  \end{enumerate}

\end{frame}

\begin{frame}{Tentative Timetable}
\vspace*{-0.4cm}                        
\begin{figure}[ht]
  \tiny
  %% if using beamer this may need :
%% \PassOptionsToPackage{table}{xcolor}
%% otherwise, the table option needs to be specified
%% for the xcolor package.

\begin{tabu}{ l l l l| l}
  %    \hline
  Week & Day & Month & Date & Description \\
  \hline
  \rowcolor{gray!25}
  34 & Wed & Aug & 24 & Course introduction \\
  \rowcolor{gray!25}
  & Thu & Aug & 25 & The central dogma (Molecular biology) \\
  35 & Wed & Aug & 31 & Introduction to Bioinformatics / Genomics \\
  & Thu & Sep & 1 & Biological databases and resources \\
  \rowcolor{gray!25}
  36 & Wed & Sep & 7 & leave \\
  \rowcolor{gray!25}
  & Thu & Sep & 8 & leave \\
  37 & Wed & Sep & 14 & leave  \\
  & Thu & Sep & 15 &  leave \\
  \rowcolor{gray!25}
  38 & Wed & Sep & 21 & Pairwise sequence alignment \\
  \rowcolor{gray!25}
  & Thu & Sep & 22 & Multiple sequence alignment \\
  39 & Wed & Sep & 28 & Database search algorithms  \\
  & Thu & Sep & 29 &  Computers: hardware, operating systems, networks and applications \\
  \rowcolor{gray!25}
  40 & Wed & Oct & 5 & General introduction to writing scripts \& programs \\
  \rowcolor{gray!25}
  & Thu & Oct & 6 & Introduction to Perl \\
  \rowfont{\color{blue!75}}
  \rowcolor{gray!25}   
  & Fri & Oct & 7 & Practical getting Perl to run (13:15-16:00) \\
  41 & Wed & Oct & 12 & Implementing an algorithm using Perl \\
  & Thu & Oct & 13 & Practical: Implementing sequence alignment in Perl \\
  \rowfont{\color{blue!75}}
  & Fri & Oct & 14 & Practical: Implemeting sequence alignment in Perl (13:15-16:00) \\
  \rowcolor{gray!25}
  42 & Wed & Oct & 19 & Sequence analysis using online \& offline tools (1) \\
  \rowcolor{gray!25}
  & Thu & Oct & 20 & Sequence analysis using online \& offline tools (2) \\
  \rowfont{\color{blue!75}}
  \rowcolor{gray!25}
  & Fri & Oct & 21 & Practical: (to be decided) (13:15-16:00) \\
  43 & Wed & Oct & 26 & Introduction to R \\
  & Thu & Oct & 27 & Summarising number collections and deriving statistics \\
  \rowcolor{gray!25}
  44 & Wed & Nov & 2 & Visualising and analysing large data sets (1) \\
  \rowcolor{gray!25}
  & Thu & Nov & 3 & Visualising and analysing large data sets (2) \\
  45 & Wed & Nov & 9 & Statistical considerations for big data ($\thicksim$ omics) \\
  & Thu & Nov & 10 & Data structure and databases \\
  \rowfont{\color{blue!75}}
  & Fri & Nov & 11 & Practical: using R to look at data. (13:15-16:00) \\
\end{tabu}

  \end{figure}
  \vspace*{+0.03cm}
  {\tiny
  All times are from 12:15 - 14:00 unless otherwise noted.
  \url{https://no.timeedit.net/web/uin/db1/engelsk/riqwQQYZ5ZXZ56Qy717Q37706gZQ16q7o5Y0756Q7Yo7.html}
  }
  
\end{frame}

\end{document}
